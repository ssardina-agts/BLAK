%%%%%%%%%%%%%%%%%%%%%%%%%%%%%%%%%%%%%%%%%%%%%%%%%%%%
\section{Discussion and Conclusion}\label{sec:discussion}
%%%%%%%%%%%%%%%%%%%%%%%%%%%%%%%%%%%%%%%%%%%%%%%%%%%%

In this paper, we proposed a technique to enhance the typical plan selection
mechanism in BDI agent systems by allowing agents to learn and adapt the context
condition of existing plans in the agent's plan library.
As defining adequate context conditions for plans may be a difficult, and
sometimes impossible, task, a framework that allows the \emph{refinement} of
(initial) context conditions would allow BDI agents to achieve better performance
over time. It also allows them to adapt gradually to an environment which may
have changed since the system was initially deployed.


This paper makes several contributions. Firstly we develop an extended
BDI framework using decision trees as (part of) context conditions,
and a \textit{probabilistic} plan selection function that caters for
both exploration and exploitation of plans. This provides a basis for
learning of context condition decision trees.  We propose and
empirically evaluate two approaches to this task, an aggressive
approach that records information from every execution, and a
conservative approach that records failure information only when it is
considered to be well-informed. Although each approach performs better
in some circumstances, the aggressive approach can, in some
situations, lead to total failure to learn. We then explore an
approach based on accounting for our confidence in the decision tree
during plan selection, rather than monitoring information recorded for
the decision tree. We show that this overcomes the issues associated
with the more aggressive approach when using the simpler plan
selection. Furthermore the new plan selection mechanism can be tuned to
different circumstances. We conclude that the preferred approach is to
use the aggressive approach to building the decision trees, together
with the confidence measure affecting plan selection.

Our experimentation did not consider the effects of using failure
recovery, under which alternative plans are tried upon the failure of
a plan for achieving a goal. Clearly, failure recovery would induce
much less ``pure'' execution traces, as execution of partial plans
would be represented in them. 
%
One limitation of our learning framework is that the learning is
\emph{local} to plans, in the sense that it is not possible to learn
interactions between sub-goals for a higher-level plan (e.g., learn
that goal $G_b$ may \textit{only} succeed if goal $G_a$ is achieved in
a certain manner). We are currently working on an approach to deal
with this issue.
%
We are also planning to do further experimentation, in particular with some more realistic
programs from existing applications. 


\bigskip
\textbf{Things to be discussed:}
\begin{enumerate}
  \item handling of continous variables (e.g., temperature)
  \item more realistic stochastic actions (easy)
  \item a way of relating worlds and learning their correlations, maybe
  could help realize when we have changed environment. can we use learning of
  bayesian networks for this?
  
\item One critique of the coverage-based confidence measure is that it has a
defined start ($c_T(S)=0$) and end state ($c_T(S)=1$). For a real
system however, learning and re-learning will occur in an endless loop
as the agent continually tries to adapt to a changing
environment. This implies that our confidence in a \dt's
classification would also require calibration based on a changing
environment. If the change in the environment is deliberate (for
instance the agent was moved into a new environment) then the coverage
$c_T(S)$ could be reset and confidence \textit{re-built}. Without such
a signal the agent must rely on measuring per-goal performance in
order to pick up failures that could be attributed to environmental
changes. The problem of identifying changes that warrant new learning
as compared to adapting previous learning, however, is generally
considered to be very difficult.
%a much harder one. 
   \item  
\end{enumerate}

%\bigskip\bigskip\bigskip

%Since decision trees can handle variables with more two values and
%even continous ones, our results generalize directly to such cases. In addition
%the results generalize trivially to scenarios where different actions can fail
%or succeed with different probabilities.
