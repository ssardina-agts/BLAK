%%%%%%%%%%%%%%%%%%%%%%%%%%%%%%%%%%%%%%%%%%%%%%%%%%%%
\section{Discussion}\label{sec:discussion}
%%%%%%%%%%%%%%%%%%%%%%%%%%%%%%%%%%%%%%%%%%%%%%%%%%%%

In this paper, we proposed a technique to enhance the typical plan selection mechanism
in BDI agent systems by allowing agents to learn and adapt the context condition of existing plans
in the agent's plan library.
%
As defining adequate context conditions for plans may be a difficult, and
sometimes 
impossible, task, a framework that allows the \emph{refinement} of (initial) context
conditions would allow BDI agents to achieve better performance over time.
%
The contributions of this paper are twofold. First, we develop an extended BDI framework in which
learning mechanisms can be used. This is done using decision trees as (part of) context
conditions, and developing a \textit{probabilistic} plan selection function that caters for both
exploration and exploitation of plans. Second, we propose and empirically evaluate two approaches
for learning such decision trees based on agent executions. We found that, in general, the more
conservative approach \BUL, which learns from failed execution traces only when decisions were
``substantially'' informed, yields more robust agent performance than the simpler aggressive
approach \CL which records every execution experience.


Our evaluation results were based on some important simplifying assumptions. In particular,
we have not considered the effects of using failure recovery, under which alternative plans are
tried upon the failure of a plan for a achieving a goal. Clearly, failure recovery would induce much
less ``pure'' execution traces, as execution of partial plans would be represented in them. Also,
we have focused on plan executability as the only objective. In some context, though, agents 
should also consider the \textit{utility} of plans and expect to learn how to achieve better utility
overall.
%
One limitation of our learning framework is that the learning is \emph{local} to plans, in the
sense that it is not possible to learn interactions between sub-goals for a higher-level plan (e.g.,
learn that goal $G_b$ may \textit{only} succeed if goal $G_a$ is achieved in a certain manner).
%
Finally, we are planning to do further experimentation, in particular with some more realistic
programs from existing applications. 

