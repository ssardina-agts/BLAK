%%%%%%%%%%%%%%%%%%%%%%%%%%%%%%%%%%%%%%%%%%%%%%%%%%%%
\section{BDI Programming}\label{sec:preliminaries}
%%%%%%%%%%%%%%%%%%%%%%%%%%%%%%%%%%%%%%%%%%%%%%%%%%%%

BDI agent-oriented programming is a popular, well-studied, and practical paradigm
for building intelligent agents situated in complex and dynamic environments with
(soft) real-time reasoning and control requirements
\cite{Benfield:AAMAS06,Georgeff89-PRS}.
% %
A BDI-style agent system consists, basically, of a \emph{belief base} (the
agent's knowledge about the world), a set of recorded pending \emph{events} or
\emph{goals}, a plan library (the typical operational procedures of the domain),
and an intention base (the plans that the agent has already committed to and is
executing).



The basic reactive goal-oriented behavior of BDI systems involves the system
responding to events---the inputs to the system---by committing to handle one
pending event-goal, \textit{selecting a plan from the library}, and placing its
program into the intention base.
% %%
A \emph{plan} in the plan library is a rule of the form $e: \psi \leftarrow
\delta$: program $\delta$ is a reasonable strategy to resolve event-goal $e$
whenever the context condition $\psi$ is believed true.
% %
Among other operations, program $\delta$ typically includes the execution of
\emph{primitive actions} ($act$) in the environment and the ``posting" of new
\emph{subgoal events} ($!e$) that ought to be resolved by selecting (other)
suitable plans.
% %
A plan may be selected for addressing an event if it is \textit{relevant} and
\textit{applicable}, that is, if it is a plan designed for the event in question
and its context condition is believed true, respectively.
% %
In contrast with traditional planning, execution happens at each step. The
assumption is that the use of plans' context-preconditions to make choices as
late as possible, together with the built-in goal-failure mechanisms, ensures
that a successful execution will eventually be obtained while the system is
sufficiently responsive to changes in the environment.



% In a BDI-style system, an agent consists, basically, of a belief base (akin to a
% database), a set of recorded pending goal events, a plan library, and an
% intention base. While the belief base encodes the agent's knowledge about the
% world, the pending events stand for the \emph{goals} the agent wants to
% achieve/resolve.
% % %
% The \textit{plan library}, in turn, contains \emph{plan rules}, or simply
% \emph{plans}, of the general form $e: \psi \leftarrow \delta$ encoding the
% standard domain operational procedure $\delta$ (that is, a program) for achieving
% the event-goal $e$ when the so-called \textit{context condition} $\psi$ is
% believed true---program $\delta$ is a reasonable strategy to resolve event $e$
% whenever $\psi$ holds. Among other operations, the plan-body program $\delta$
% typically includes the execution of actions ($act$) in the environment and
% subgoal events ($!e$) that ought to be resolved by selecting suitable plans.
% Lastly, the intention base accounts for the current, partially instantiated,
% plans that the agent has already committed to and is executing.


% The basic reactive goal-oriented behavior of BDI systems involves the system
% responding to events, the inputs to the system, by committing to handle one
% pending event-goal, \textit{selecting a plan} from the library, and placing its
% program body  into the intention base.
% % %%
% A plan may be selected if it is \textit{relevant} and \textit{applicable}, that
% is, if it is a plan designed for the event in question and its context condition
% is believed true, respectively.
% % %
% In contrast with traditional planning, execution happens at each step. The
% assumption is that the use of plans' context-preconditions to make choices as
% late as possible, together with the built-in goal-failure mechanisms, ensures
% that a successful execution will eventually be obtained while the system is
% sufficiently responsive to changes in the environment.


% For the purposes of this paper, we shall mostly focus on the plan library, as we
% investigate ways of learning how agents can make a better use of it over time.
For the purposes of this paper, we shall mostly focus on the plan library.
% %
It is not hard to see that, by grouping together plans responding to the same
event type, the plan library can be seen as a set of \emph{goal-plan tree}
templates: a goal (or event) node has children representing the alternative
relevant plans for achieving it; and a plan node, in turn, has children nodes
representing the subgoals (including primitive actions) of the plan.
% %%
These structures, can be seen as AND/OR trees: for a plan to succeed all the
subgoals and actions of the plan must be successful (AND); for a subgoal to
succeed one of the plans to achieve it must succeed (OR).

Consider, for instance, the goal-plan tree structure depicted in
Figure~\ref{fig:T3}.
% %%
A link from a goal to a plan means that this plan is relevant (i.e., potentially
suitable) for achieving the goal (e.g., $P_1 \ldots P4$ are the relevant plans
for event-goal $G$); whereas a link from a plan to a goal means that the plan
needs to achieve that goal as part of its (sequential) execution (e.g., plan
$P_A$ needs to achieve goal $G_{A1}$ first and then $G_{A2}$).
% %
For compactness, an edge with a label $\times n$ states that there are $n$ edges
of such type.
% %
Leaf plans directly interact with the environment and so, in a given world state,
they can either succeed or fail when executed; this is marked accordingly in the
figure \emph{for some particular world} (of course such plans may behave
differently in other states).
% %
In some world, given successful completion of $G_A$ first, the agent may achieve
goal $G_B$ by selecting and executing $P_B$, followed by selecting and executing
$2$ leaf working plans to resolve goals $G_{B1}$ and $G_{B2}$. If the agent
succeeds with goals $G_{B1}$ and $G_{B2}$, then it succeeds for plan $P_B$,
achieving thus goal $G_B$ and the top-level goal $G$ itself. There is no possible
successful execution, though, if the agent decides to carry on any of the three
plans labelled $P_{B2}'$ for achieving low-level goal $G_{B2}$.




Clearly, the problem of \textit{plan-selection} is at the core of the BDI
approach:
\emph{which plan should the agent commit to in order to achieve a certain goal?}
% %
This problem amounts, at least partly, to what has been referred to as
\emph{means-end analysis} in the agent foundational literature
\cite{Pollack92-IRMA,Bratman88}, i.e., the decision of \textit{how}
goals are achieved.
% %%
To tackle the plan-selection task, state-of-the-art BDI systems leverage domain
expertise by means of the context conditions of plans. However, crafting fully
correct context conditions at design-time can be a
% correct context conditions at design-time can, in some applications, be a
demanding and error-prone task. Also, fixed context conditions do not
allow agents to adapt to changing environments.
% %%
Below, we shall provide an extended BDI framework that
allows agents to learn/adapt plans' context conditions, and discuss and empirically
evaluate different approaches for such learning task.


% 
% Interestingly, the structured information contained in the goal plan tree can
% also be used to provide guidance to the learning module. In particular, consider
% the context condition of plans, which are critical for guiding the execution of
% the agent program. A plan will not be used in the current state if its context
% condition is not satisfied.  Incorrect or inadequate context conditions can lead
% to two types of problems.  If the context condition of a plan is
% \textit{over-constrained} and evaluates to false in some situations where the
% plan could succeed then this plan will simply never be tried in those
% situations, resulting in possible utility loss to the agent.  On the other hand,
% if the context condition is \textit{under-specified}, it may evaluate to true in
% some situations where the plan will be ineffective.  Such ``false triggers''
% will result in unnecessary failures, and although the agent may recover by
% choosing alternative plans, it may lose valuable time or waste resources
% unnecessarily, thereby losing utility.  Hence, it would be preferable to learn
% from experience to avoid using plans that are unlikely to succeed at particular
% environmental states.
% 
% The rest of this paper explores the issue of learning to improve the
% context conditions specified by the programmer.


