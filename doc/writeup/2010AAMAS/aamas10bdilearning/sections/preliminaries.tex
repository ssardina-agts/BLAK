%%%%%%%%%%%%%%%%%%%%%%%%%%%%%%%%%%%%%%%%%%%%%%%%%%%%
% \section{Preliminaries}\label{sec:preliminaries}
\section{BDI Programming}\label{sec:preliminaries}
%%%%%%%%%%%%%%%%%%%%%%%%%%%%%%%%%%%%%%%%%%%%%%%%%%%%

BDI agent-oriented programming is a popular, well-studied, and practical paradigm for
building intelligent agents situated in complex and dynamic environments with
(soft) real-time reasoning and control requirements \cite{Georgeff89-PRS}.
%
BDI systems enable \emph{abstract plans} written by programmers to be combined and used in
real-time, in a way that is both flexible and robust. 
%
There are a plethora of agent programming languages and development
platforms in the BDI tradition, such as  \PRS\ \cite{Georgeff89-PRS},
\JACK~\cite{BRHL98}, \TAPL~\cite{Hindriks99:Agent}, and \JASON~\cite{jasonbook}, among others.


In a BDI-style system, an agent consists, basically, of a belief base (akin
to a database), a set of recorded pending events, a plan library, and an
intention base. While the belief base encodes the agent's knowledge about the world, the
pending events stand for the \emph{goals} the agent wants to achieve/resolve. 
%
\notem{fix notation for plans, plan rule should be $P$}
The \textit{plan library}, in turn, contains \emph{plan rules} of the general 
form $e: \psi \leftarrow P$ encoding the standard domain operation-strategy $P$
(that is, a program) for achieving the event-goal $e$ when so-called \textit{context condition}
$\psi$ is believed true---program $P$ is a reasonable strategy to resolve event $e$ whenever $\psi$
is believed to hold.
%
The plan-body program $P$ typically contains actions ($act$) to be performed in the environment,
belief updates ($+b$ and $-b$ to add or delete a proposition, respectively), tests ($?\phi$)
operations, and subgoals ($!e$) to post internal subgoal-events which ought to be in turn resolved
by selecting suitable plans for that event.
%
Lastly, the intention base accounts for the current, partially instantiated, plans
that the agent has already committed to in order to handle or achieve some event-goal.


The basic reactive goal-oriented behavior of BDI systems involves the system responding to
events, the inputs to the system, by committing to handle one pending event-goal, \textit{selecting
a plan} from the library, and placing its program body  into the intention base.
%
A plan may be selected if it is \textit{relevant}, that is, it is a plan designed for the even in
question, and it is \emph{applicable}, that is, its context condition is believed true.
%
In contrast with traditional planning, execution happens at each step. The assumption is that
the use of plans' context-preconditions to make choices as late as possible, together with
the built-in goal-failure mechanisms, ensures that a successful execution will eventually be
obtained while the system is sufficiently responsive to changes in the environment.


For the purposes of this paper, we shall mostly focus on the plan library, as we investigate ways of
learning how agents can make a better use of it over time.
%
It is not hard to see that, by grouping together plans responding to the same event type, the
plan library can be seen as a set of \emph{goal-plan tree} templates: a goal (or event) node has
children representing the alternative relevant plans for achieving it; and a plan node,
in turn, has children nodes representing the subgoals (including primitive actions) of the plan. 
%
These structures, can be seen as AND/OR trees: for a plan to succeed all the subgoals and
actions of the plan must be successful (AND); for a subgoal to succeed one of the plans to
achieve it must succeed (OR).

Consider, for instance, the goal-plan tree structure depicted in Figure~\ref{fig:T3}.
%
A link from a goal to a plan means that this plan is relevant (i.e., potentially suitable)
for achieving the goal (e.g., $P_1,P_1',P_2',P_3'$, and $P_4'$ are the relevant plans for
$G$); whereas a link from a plan to a goal means that the plan needs to achieve that goal as part of
its (sequential) execution (e.g., $P_1$ needs to achieve first goal $G_1$ and then $G_2$). 
%
An edge with a label $\times n$ states that there are $n$ edges of such type. 
% Indexes are used to represent different goals/plans under such labeled edges (e.g., the indexes
% below $P_1$ are understood as $i \in \{1,2\}$, $j \in \{1,\ldots,3\}$, and $k \in
% \{2,\ldots,5\}$).
%
Leaf plans are assumed to directly succeed or fail when executed in the environment, and are
marked accordingly in the figure for some particular world.
%
In some world, for our example, the agent may achieve goal $G_2$ by
selecting and executing $P_1^2$ followed by selecting and executing $3$
working plans, namely, $P_1^a$, $P_1^b$, and $P_1^c$ to resolve goals $G_a$, $G_b$, and $G_c$,
respectively. An analogous situation arises below plan $P_1^1$, but it is abstracted out in
the figure for legibility. If the agent succeeds with goals $G_1$ and
$G_2$, then it succeeds for plan
$P_1$, achieving thus the top-level goal. There is no possible successful execution, though, if the
agent decides to carry on any of the four plans $P_i'$, $i \in \{1,\ldots,4\}$, instead for
achieving the top-level goal $G$.


The problem of \textit{plan-selection} is at the core of the
whole BDI approach: \emph{which plan should the agent commit to in order to achieve a certain
goal?}
%
This problem amounts, at least partly, to what has been referred to as \emph{means-end analysis} in the
agent foundational literature \cite{Pollack92-IRMA,Bratman88}, that is, the decision of \textit{how}
goals are achieved.
%
To tackle the plan-selection task, BDI systems leverage domain
expertise by means of the 
context conditions of plans. However, crafting fully correct context conditions at design-time can,
in some applications, be a demanding and error-prone task. In addition, fixed context conditions do
not allow agents to adapt to new environments. 
%
% Very little work has been devoted to extend the paradigm to richer accounts of plan-selection,
% with the notable exception of \cite{Bordini02} that uses .......
%
In the rest of the paper, we shall provide an extended framework for BDI agent systems that allows
agents to learn/adapt plans' context conditions, and discuss and evaluate two possible approaches
for such learning task.







% 
% Interestingly, the structured information contained in the goal plan tree can
% also be used to provide guidance to the learning module. In particular, consider
% the context condition of plans, which are critical for guiding the execution of
% the agent program. A plan will not be used in the current state if its context
% condition is not satisfied.  Incorrect or inadequate context conditions can lead
% to two types of problems.  If the context condition of a plan is
% \textit{over-constrained} and evaluates to false in some situations where the
% plan could succeed then this plan will simply never be tried in those
% situations, resulting in possible utility loss to the agent.  On the other hand,
% if the context condition is \textit{under-specified}, it may evaluate to true in
% some situations where the plan will be ineffective.  Such ``false triggers''
% will result in unnecessary failures, and although the agent may recover by
% choosing alternative plans, it may lose valuable time or waste resources
% unnecessarily, thereby losing utility.  Hence, it would be preferable to learn
% from experience to avoid using plans that are unlikely to succeed at particular
% environmental states.
% 
% The rest of this paper explores the issue of learning to improve the
% context conditions specified by the programmer.


