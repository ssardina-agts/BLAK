%%%%%%%%%%%%%%%%%%%%%%%%%%%%%%%%%%%%%%%%%%%%%%%%%%%%
% \section{Two Approaches to Context Learning}\label{sec:context_learning}
\section{Context Learning: 2 Approaches}\label{sec:context_learning}
%%%%%%%%%%%%%%%%%%%%%%%%%%%%%%%%%%%%%%%%%%%%%%%%%%%%

\newcommand{\success}{\mbox{\emph{succ}}}
\newcommand{\failure}{\mbox{\emph{fail}}}

With the classical BDI programming framework extended with decision trees and a
probabilistic plan selection scheme, we are now ready to develop mechanisms for
learning context decision trees
%and hopefully improving over time the selection
%of plans based on experience.
based on online experiences, in order to improve plan selection over time.
% %
To that end, in this section, we explore two approaches for learning the context
condition of plans.



Recall that our objective is to learn which plans are best for achieving
a particular goal event in the various world states that may ensue. Given that,
in this work, we have no measure of cost for plans,\footnote{This could also be a
useful addition, but is not part of standard BDI programming languages.} a good
plan for a given world state is simply one which (usually) succeeds in such
state. In order to learn the context decision tree for a plan, the agent keeps
track of previous experiences it has had when running the plan in question. More
precisely, if a plan $P$ is tried in world state $w$ with certain outcome $o \in
\{\success,\failure\}$, the agent may record the tuple $\tuple{w,o}$ as part of
the training set for plan $P$.
% %
Interestingly, while it is always meaningful to record successful executions,
some failures may not be worth recording. Based on this observation, we shall
develop and compare two different algorithms that differ on how past experiences
are taken into account by the agent. Before then, though, let us explain better
this issue by means of an example.
 

\begin{figure}[t]
\begin{center}
\begin{tikzpicture}[scale=0.8,level distance=1.1cm]
\tikzstyle{txt}=[scale=.9]
\tikzstyle{succ}=[label=below:$\surd$]
\tikzstyle{fail}=[label=below:$\times$]

\tikzstyle{planbox}=[draw,minimum height=0.55cm,minimum width=0.55cm]
\tikzstyle{goalbox}=[draw,rounded corners,minimum height=0.55cm,minimum width=0.55cm]

	
\tikzstyle{level 1}=[sibling distance=4.0cm] 
\tikzstyle{level 2}=[sibling distance=4cm] 
\tikzstyle{level 3}=[sibling distance=2cm]
\tikzstyle{level 4}=[sibling distance=2.5cm]
\tikzstyle{level 5}=[sibling distance=1.3cm]

\node[goalbox,yshift=1cm,solid] (T) {$G$}
	child[dashed] {node[planbox] (P1) {$P_1$}}
	child[solid] {node[planbox] (Pi) {$P_i$}
		child {node[goalbox] {$G_A$}
			child {node[planbox] {$P_{A}$}
				child {node[goalbox] {$G_{A1}$}
					child {node[planbox,succ] {$\phantom{P}$}}
					child {node[planbox,fail] {$\phantom{P}$} 
					edge from parent node[txt,right,near start] {$\times 3$}
					}
				}
				child {node[goalbox] {$G_{A2}$}
					child {node[planbox,succ] {$\phantom{P}$}}
					child {node[planbox,fail] {$\phantom{P}$} 
						edge from parent node[txt,right,near start] {$\times 3$}
					}
				}
			}
			child {node[planbox,fail] {$\phantom{P}$} 
				edge from parent node[txt,right,near start] {$\times 3$}
			}
		}
		child {node[goalbox] {$G_{B}$}
			child {node[planbox,fail] {$\phantom{P}$}
				edge from parent node[txt,left,near start] {$\times 3$}
			}
			child {node[planbox] {$P_B$}
				child {node[goalbox] {$G_{B1}$}
					child {node[planbox,succ] {$\phantom{P}$}}
					child {node[planbox,fail] {$\phantom{P}$}
						edge from parent node[txt,right,near start] {$\times 3$}
					}
				}
				child {node[goalbox] {$G_{B2}$}
					child {node[planbox,succ] {$P_{B2}$}}
					child {node[planbox,fail] {$P_{B2}'$}
						edge from parent node[txt,right,near start] {$\times 3$}
					}
				}
			}
		}
	}
	child[dashed] {node[planbox] (P4) {$P_4$}}
;
\node[txt] at ($ (P1)!.5!(Pi) $) {$\ldots$};
\node[txt] at ($ (Pi)!.5!(P4) $) {$\ldots$};

\end{tikzpicture}

\end{center}
\caption{Goal-plan hierarchy $\T_3$. There are $2^4$ worlds whose solutions are
distributed evenly in each of the $4$ top level plans. Successful execution
traces are of length $4$. Within each sub-tree $P_i$, \BUL\ is expected to
perform better for a given world, but it suffers in the number of worlds. Overall, \CL\ and \BUL\
perform equally well in this structure.}
\label{fig:T3}
\end{figure}



Consider the example in Figure \ref{fig:T3}.
% %
Suppose that in some execution, plan $P_i$, for some $i \in \{1,\ldots,4\}$, is
selected in order to resolve top-goal $G$ in some world state $w_1$. The plan
involves, in turn, the successful resolution of sequential goals $G_A$ and $G_B$.
Suppose further that subgoal $G_A$ has been resolved successfully, yielding new
state $w_2$, and that plan $P_B$ has been chosen next to try and achieve the
second subgoal $G_B$.
% %
Suppose next that the first subgoal of plan $P_B$, namely $G_{B1}$ has been
successfully resolved, yielding new state $w_3$, but that the non-working plan
$P_{B2}'$ for subgoal $G_{B2}$ is selected in $w_3$ and execution thus
\emph{fails}.
% %
As there is no failure recovery, this failure will be propagated upwards in the
hierarchy, causing goals $G_{B2}$ as well as $G_B$ and top-level goal $G$ itself
to fail.
% %
First of all, the failure (in world state $w_3$) must be recorded in the decision
tree of the plan where the failure originated, namely, plan $P_{B2}'$. 
%Such plan is in fact ``low-level,'' in that it has no subgoals and thus interacts with the
%external world directly. Hence, all we can expect is to learn such interaction
%over time.
Such bottom-level plans have no subgoals, so they interact with the
external world directly, and over time we can expect to learn such interactions.
% %
On the other hand it is unclear, as we will show below, whether failure should also be
recorded in the decision trees for plans higher up in the hierarchy (i.e. plans
$P_B$ and $P_i$).






In order to discuss further \emph{which} data should be recorded \emph{where}, we
define the notion of an \textit{active execution trace}, as a sequence of the
form $G_0[P_0:w_0] \cdot G_1[P_1:w_1] \cdot \ldots \cdot G_n[P_n:w_n]$, which
represents the sequence of currently active goals, along with the plans which
have been selected to achieve each of them, and the world state in which the
selection was made---plan $P_i$ has been selected in world state $w_i$ in order
to achieve the $i$-th active subgoal $G_i$.
% %
In our example, the trace at the moment of failure is as follows: \[
\lambda=G[P_i:w_1] \cdot G_B[P_B:w_2] \cdot G_{B2}[P_{B2}':w_3]. \]


So, when the final goal in $\lambda$ fails, namely $G_{B2}$, it is clear that the
decision tree of the plan being used to achieve this goal ought to be updated,
and a failure should be recorded for the world state $w_3$ against 
the decision tree attached to plan $P_{B2}'$.  
%%
By recording every outcome for the lowest plans, i.e., plans with no subgoals,
the system eventually learns how such plans perform in the environment.
%%
% Although it may be the case
% that the plan usually succeeds in the situation in which it was chosen, and failure is
% simply due to some non-determinism (or in the general case, actions of other
% agents, interactions, etc.), there is no way to determine this and the learning
% process will eventually recognise such cases as ``noise.''
% %

What is more difficult to determine is whether the decision trees of plans
associated with \emph{earlier goals} in $\lambda$ should also be updated.
% %
More concretely, should failure cases in world states $w_2$ and $w_1$ be recorded
against plans $P_B$ and $P_i$, respectively?
% %
The point is that it is conceivable that the failure of subgoal $G_{B2}$ in plan
$P_B$, for instance, could indeed have been avoided, had the alternative plan
$P_{B2}$, been chosen instead. Therefore, recording a failure against plan $P_B$
would not be justified---it is not true that plan $P_{B}$ is a ``bad'' choice
in world state $w_2$.
% %
Informally, one could argue that it is more appropiate to \emph{wait} before
recording failures against a plan until one is reasonably confident that
subsequent choices down the goal-plan tree hierarchy were ``well informed.'' In
our case, if the agent knows that the plan selection for goal $G_{B2}$ was as
good and informed as possible, then recording the failure for world $w_2$ in plan
$P_B$ would also be justified. Similarly, if the agent considers that the plan
selection for subgoal $G_B$ was an informed choice, then recording the failure
for world $w_1$ against $P_i$'s decision tree would be justified too.


\newcommand{\procedurefont}[1]{\mathsf{#1}}
\newcommand{\StableGoal}{\procedurefont{StableGoal}}
\newcommand{\RecordTrace}{\procedurefont{RecordFailedTrace}}
\newcommand{\RecordWorldDT}{\procedurefont{RecordWorldDT}}




The judgement as to whether plan choices were sufficiently ``well informed,'' is
however not a trivial one.  A failed plan $P$ is considered to be \emph{stable} for a
particular world state $w$ if the rate of success of $P$ in $w$ is changing below
a certain threshold $\epsilon$. In such a case, the agent can start to build
confidence about the applicability level of $P$. The stability notion extends to
goals as follows: a failed goal is considered \emph{stable} for world state $w$ 
if all its relevant plans are stable for $w$.
% %
When a goal is stable, we regard the plan selection for such goal as a ``well
informed'' one. Thus, a failure is recorded in the plan for a given world if the subgoal that failed is stable for the respective world in which it was resolved. In our example, we
record the failure in plan $P_B$ ($P_i$) if goal $G_{B2}$ ($G_B$) is deemed stable
in world state $w_3$ ($w_2$), that is, if the selection of option $P_{B2}'$
($P_B$) was an informed one.



The $\RecordTrace$ algorithm below shows how a failed execution run $\lambda$ is
recorded. Function
$\StableGoal(G,w,k,\epsilon)$ returns true iff goal $G$ is considered \textit{stable} for world state $w$, for success rate change 
threshold $0 < \epsilon \leq 1$ and minimal number of executions $k
\geq 0$.
%
The algorithm starts by recording the failure against the last plan $P_n$ in the
trace.
% %
Next, if the choice of executing plan $P_n$ to achieve goal $G_n$ was deemed an
informed one (that is, goal $G_n$ was stable for $w_n$), then the procedure
should be repeated for the previous goal-plan nodes, if any.
% %
If, on the other hand, the last goal $G_n$ in the trace is not considered stable
enough, the procedure terminates and no more failure data is assimilated.
% %
Observe that, in order to update the decision tree of a certain plan that was
chosen along the execution, it has to be the case that the (failed) decisions
taken during execution must have all been informed ones.
Note that the stability idea only applies to failures since successes are always recorded.
 \renewcommand{\algorithmiccomment}[1]{\hfill \texttt{\small // #1}}
 \newcommand{\assign}{\mbox{:=\ }}
 \begin{algorithm}[h]
	\caption{$\RecordTrace(\lambda,k,\epsilon)$}\label{algo:record_failed_exec}
	\label{alg:NDS}
  \begin{algorithmic}[1]
    \REQUIRE $\lambda=G_0[P_0:w_0] \cdot \ldots \cdot G_n[P_n:w_n]$; $k\geq0$;
    $\epsilon > 0$ \ENSURE Propagates DT updates for plans

	\STATE $\RecordWorldDT(P_n,w_n,\failure)$

    \IF{$\StableGoal(G_n,w_n,k,\epsilon) \land |\lambda|>1$}
    	 \STATE $\lambda' \assign G_0[P_0:w_0] \cdot \ldots \cdot
    				G_{n-1}[P_{n-1}:w_{n-1}]$
    	\STATE $\RecordTrace(\lambda',k,\epsilon)$ \COMMENT{recursive call}
    \ENDIF
  \end{algorithmic}
\end{algorithm}

 



So, in the remainder of the paper, we shall consider two learning approaches
compatible with the framework developed in the previous section. The first, which
we call \emph{aggressive concurrent learning} (\CL), corresponds to the more
traditional approach where all data is always assimilated by the learner, that
is, we take $\epsilon = 1$ and $k = 0$. In other words, every plan and every goal
is always considered stable and, as a result, a failure in a plan is always
recorded. The assumption is that misleading information, as discussed above, will
eventually disappear as noise.
% %
The second one, which we refer to as \emph{bottom-up learning} (\BUL), is more
cautious and records a failure execution experience when the success rate has stabilised i.e. is not changing by more than $\epsilon$.
In our work, we have taken
$\epsilon = 0.3$ and $k = 3$, that is, the context condition of a plan is
considered stable (for a world state) if at least $3$ past execution experiences
have been recorded 
%and the rate of success has lately been changing less than $0.3$.
and the change in the rate of success over the last two experiences is less than $0.3$.
% %
Note that the lower $\epsilon$ is and the higher $k$ is, the more conservative
the agent is in considering its decisions ``well informed.''

In the following section, we shall explore these two approaches against
different programs with different structures.




