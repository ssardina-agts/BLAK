%%%%%%%%%%%%%%%%%%%%%%%%%%%%%%%%%%%%%%%%%%%%%%%%%%%%
\section{Context Learning}\label{sec:context_learning}
%%%%%%%%%%%%%%%%%%%%%%%%%%%%%%%%%%%%%%%%%%%%%%%%%%%%


With the classical BDI programming framework extended with decision
trees and a probabilistic plan selection scheme, we are now ready to
develop mechanisms for improving the selection of plans, based on
experience.
%
To that end, in this section, we develop two approaches for learning
the context condition of plans.

What we are aiming to learn is which plans are best selected for
achieving a particular goal, in the various world states that
occur. Given that we have no measure of cost of particular
plans,\footnote{This could also be a useful addition, but is not part
  of standard BDI programming languages.} a good plan for a
particular world state is simply one which (usually) succeeds in that
situation. Thus we need to store in our decision trees information
about world states in which plans succeeded and failed.  However, as
we will show with an example, while it is always meaningful to record
success, some failures may be better not recorded.

\begin{figure}[t]
\begin{center}
\begin{tikzpicture}
[
level distance=27mm, 
level 1/.style={sibling distance=20mm}, 
level 2/.style={sibling distance=12mm}, 
level 3/.style={sibling distance=12mm},
goal/.style={ellipse, very thin, draw=red!20, top color=white, bottom color=red!20, inner ysep=2mm, text width=15mm, text centered, font=\itshape},
plan/.style={rectangle, very thin, draw= black!20, top color=white, bottom color=black!20, inner ysep=3mm, text width=19mm, text centered}
] 
\node[goal] at (0,0) {TopGoal} [grow=right]
	child {node[plan] {Charge From Home Driveway}}
	child {node[plan] {Charge From Home Solar}}
	child {node[plan] {Supply Power To Home}} 
	child {node[plan] {Charge From Shops Grid}}
	child {node[plan] {Supply Power To Shops}}
	child {node[plan] {Charge From Office Grid}}
	child {node[plan] {Charge From Office Wind Power}}
	child {node[plan] {Supply Power To Office}
}; 
\node[goal]  at (5.5,5.5) {Connect} [grow=right]
	child {node[plan] {Connect To Driveway}}
	child {node[plan] {Connect To Home Solar}}
	child {node[plan] {Connect To Shops Grid}}
	child {node[plan] {Connect To Office Wind Power}}
	child {node[plan] {Connect To Office Grid}
};

\node[goal]  at (11,4.5) {Charge} [grow=right]
	child {node[plan] {Charge To Level 025}}
	child {node[plan] {Charge To Level 050}}
	child {node[plan] {Charge To Level 075}}
	child {node[plan] {Charge To Level 100}
};

\node[goal]  at (11,-4.5) {Discharge} [grow=right]
	child {node[plan] {Discharge To Level 000}}
	child {node[plan] {Discharge To Level 025}}
	child {node[plan] {Discharge To Level 050}}
	child {node[plan] {Discharge To Level 075}
};

\node[goal]  at (5.5,-5.5) {Disconnect} [grow=right]
	child {node[plan] {Disconnect From Driveway}}
	child {node[plan] {Disconnect From Home Solar}}
	child {node[plan] {Disconnect From Shops Grid}}
	child {node[plan] {Disconnect From Office Wind Power}}
	child {node[plan] {Disconnect From Office Grid}
};
\end{tikzpicture}

\end{center}
% \caption{Goal-plan tree hierarchy. Rounded boxes stand for event goals; rectangles for plans.}
\caption{Goal-plan tree hierarchy $\T_3$. Rounded boxes stand for event goals; rectangles for
plans.}
%
% \caption{Goal-plan tree hierarchy $\T_3$. Rounded boxes stand for event goals; rectangles for
% plans. Leaf plans are assumed to directly succeed or fail when executed in the environment, and
% are marked accordingly. 
% %
% An edge with a label $\times n$ states that there are $n$ edges of such type (e.g.,  goal $G$ has
% $5$ relevant plans). Indexes are used to represent different goals/plans under such labeled edges.
% For instance, below $P_1$, the indexes are understood as $i \in \{1,2\}$, $j \in \{1,\ldots,3\}$,
% and $k \in \{2,\ldots,5\}$.
% %
% To succeed, an agent should execute $6$ working plans, namely, $P_{1111}^1$, $P_{1112}^1$, and 
% $P_{1113}^1$ to resolve goal $G_1^1$, and $P_{1211}^1$, $P_{1212}^1$, and 
% $P_{1213}^1$ to resolve goal $G_1^2$.}
\label{fig:tree03}
\end{figure}

Consider the example in Figure \ref{fig:tree03}. 
%
Suppose that in some execution, plan $P_1$ is chosen to address goal $G$, in some world state $w_1$;
that subgoal $G_1$ is successfully executed; and that plan $P_1^2$ is chosen to try and achieve
the second subgoal $G_2$, requiring the achievement of $G_a$, $G_b$ and $G_c$.
%
Suppose further that $G_a$ is successfully achieved using $P_1^a$, but execution \emph{fails}
when trying plan $P_3^b$ for achieving subgoal $G_b$.  As we have no failure recovery, this will
immediately cause $G_b$ to fail, and recursively will cause failure at each stage. 
%
Clearly, the failure should be recorded in the decision tree of the plan that caused the
failure, namely, plan $P_3^b$. However, it is not so clear, as will be discussed next, whether the
failure should be recorded in the decision trees for plans higher up in the hierarchy, that is,
plan $P_1^2$ and even top-level plan $P_1$.

% Consider the example shown in Figure \ref{fig:gptree}: a link from a
% goal to a plan means that this plan is relevant (potentially suitable)
% for achieving the goal (e.g. $P_2$ is a relevant plan for $G_0$). A
% link from a plan to a goal means that the plan needs to achieve that
% goal as part of its execution (e.g. $P_1$ needs to achieve $G_1$ and
% then $G_2$).  Suppose that in some execution, plan $P$ is chosen to
% address goal $G$, in some world state $w$, and $G_0$ is the first step
% in plan $P$. Suppose further that plan $P_1$ is chosen to try and
% achieve $G_0$, requiring achievement of $G_1$, $G_2$ and $G_3$; $G_1$
% is successfully achieved using, say, $P_3$, but after choosing $P_5$
% to achieve $G_2$, there is a failure in $P_5$.  As we have no failure
% recovery, this will immediately cause $G_2$ to fail, and recursively
% will cause failure at each stage. Clearly the failure at the bottom of
% the tree ($P_5$) should be recorded in the decision tree data for that
% plan. However it is not so clear, as will be discussed next, whether
% the failure should be recorded in the decision trees for $P_1$ and
% $P$.


In order to discuss further which data should be recorded where, we
define the notion of an \textit{active execution trace} of the form
$G_0[P_0:w_0] \cdot G_1[P_1:w_1] \cdot \ldots \cdot G_n[P_n:w_n]$, which
represents the sequence of currently active goals, along with the plans which 
have been selected to achieve each of them, and the world state in which the selection 
was made---plan $P_i$ has been selected in world state $w_i$ in order to achieve the $i$-th
active subgoal $G_i$.
%
In our example, the trace at the moment of failure would be the following one:
\[
\lambda=G[P_1:w_1] \cdot G_2[P_1^2:w_2] \cdot G_b[P_3^b:w_3].
\]

When the final goal of $\lambda$ fails, it is clear that the decision
tree of the plan being used to achieve that goal ought to be updated, in
the example, a failure should be recorded for the world state $w_3$
for the decision tree attached to plan $P_3^b$.  Although it may be the case
that the plan usually succeeds in the situation in which it was
chosen, and failure is simply due to some non-determinism (or in the
general case, actions of other agents, interactions, etc.), there is
no way to determine this and the learning process will eventually
recognise such cases as ``noise.'' However, it is less clear whether
the decision trees of plans associated with earlier goals in $\lambda$
should be updated.  It may be, in fact, that the failure of $G_b$ could
have been avoided, had the alternative plan $P_1^b$ been chosen instead.
If this is the case, then recording a failure against $P_1^2$ and $P_1$ is not
justified.  It may therefore be better to wait before recording failures
against a plan until one is reasonably confident that subsequent choices were ``well informed.''

\newcommand{\procedurefont}[1]{\mathsf{#1}}
\newcommand{\StableGoal}{\procedurefont{StableGoal}}
\newcommand{\RecordTrace}{\procedurefont{RecordFailedTrace}}
\newcommand{\RecordWorldDT}{\procedurefont{RecordWorldDT}}

% The judgment as to whether the decisions made were sufficiently ``well
% informed'', is however not a trivial one. We develop an algorithm that
% makes the decision as to whether to record a failure, based on a
% notion of stability, where a plan is considered to be stable for a
% particular world state $w$ if the rate of success of $P$ in $w$ is
% changing below a certain threshold, $\epsilon$. We also allow
% specification of a minimum number of execution experiences for $P$ in
% $w$, in order to have the change of rate of success be sufficiently
% meaningful. A goal is considered to be stable for world state $w$ if
% all its relevant plans are stable for $w$.

% Stephane replaced the paragraph. I thought it was better to present
% the notion of stability first, then to say that stability is what we
% consider as well informed, and then introduce the algorithm.

The judgment as to whether the decisions made were sufficiently ``well
informed,'' is however not a trivial one.  A plan $P$ is considered to
be \emph{stable} for a particular world state $w$ if the rate of
success of $P$ in $w$ is changing below a certain threshold,
$\epsilon$.
% comments by Stephane 
% o I guess here, something more correct would be that the rate of
% success of P in the leaf node of the decision tree that contains w.
% I'm not sure it is worth precising.
% o is the rate of change computing over a window of the last n use of
% the plan?
%stephane adds:
If the rate of success does not change much after some
observations, we can start to build confidence about it.
% end add
We also allow specification of a minimum number of execution
experiences for $P$ in $w$, in order to have the change of rate of
success be sufficiently meaningful. A goal is considered to be stable
for world state $w$ if all its relevant plans are stable for $w$.  We
consider that the decision of recording a failure for a plan is ``well
informed'' when the goal it is being chosen for is stable.  We use a function
$\StableGoal(G,w,k,\epsilon)$ which returns true iff goal $G$ is
considered \textit{stable} for world state $w$, under minimal
% stephane changed from minimal execution traces to number of
% execution, I was getting confused whether there was a concpet of
% "minimal trace". I thought it would be clearer like this.
 number of executions and change of success rate thresholds $k \geq 0$
 and $0 < \epsilon \leq 1$, respectively.

\begin{algorithm}\caption{$\RecordTrace(\lambda,k,\epsilon)$}\label{algo:record_failed_exec}
KwData{$\lambda=G_0[P_0:w_0] \cdot \ldots \cdot G_n[P_n:w_n]$; $k\geq0$; $\epsilon > 0$}
\medskip

$\RecordWorldDT(P_n,w_n,failure)$

If{$[\StableGoal(G_n,w_n,k,\epsilon) \land |\lambda|>1]$} {
	tcp{\small decision for $G_n$ was informed}
	$\lambda' \longleftarrow G_0[P_0:w_0] \cdot \ldots \cdot G_{n-1}[P_{n-1}:w_{n-1}]$
	$\RecordTrace(\lambda',k,\epsilon)$
}
\end{algorithm}

The algorithm starts by assimilating the failure in the last plan
$P_n$ in the trace, by recording the world $w_n$ in which $P_n$ was
started as a ``failure'' case.
% Stephane modified here to mention that there must be a previous node
% here, which explains the |\lambda| > 1
If there is a previous node in the trace (i.e. for the plan $P_{n-1}$
which required achievement of the failed goal $G_n$) and the choice of
executing plan $P_n$ to achieve goal $G_n$ was indeed an informed one
(that is, goal $G_n$ was stable for $w_n$), the procedure is repeated
for the previous node in the trace.  If, on the other hand, the last
goal $G_n$ in the trace is not considered stable enough, the procedure
terminates and no more data is assimilated.
%
It follows then that, in order to update the decision tree of a certain
plan, it has to be the case that the (failed) decisions taken during
the plan execution must have been informed enough.
The closer to 0 $\epsilon$ is, and the higher $k$ is, the more
conservative the agent will be in considering its decisions well
informed. With $\epsilon$ = 1 and k = 0 we obtain a more standard
learning approach where all information is accumulated, and the
assumption is made that faulty information will eventually disappear
as noise.

So, in the remaining of the paper, we shall compare two cases. The first we call
aggressive concurrent learning (\CL) and is exactly the more traditional approach where all data is
assimilated by the learner, that is, we take $\epsilon = 1$ and $k = 0$.
% \footnote{Those values imply that plans/goals are always deemed stable.}
%Stephane questions: is it really k=0? Shouldn'it be at least 1?
The second we call bottom-up learning (\BUL), and we use $\epsilon = 0.3$ and
$k = 3$.
%according to Scott's email.
We explore these two approaches on some programs with different
structures  and discuss the results.
