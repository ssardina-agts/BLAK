%%%%%%%%%%%%%%%%%%%%%%%%%%%%%%%%%%%%%%%%%%%%%%%%%%%%
\section{Introduction}\label{sec:intro}
%%%%%%%%%%%%%%%%%%%%%%%%%%%%%%%%%%%%%%%%%%%%%%%%%%%%



It is widely believed that learning is a key aspect of intelligence, as
it enables adaptation to complex and changing environments.  
%
Agents developed under the Belief-Desire-Intention (BDI)
approach~\cite{Bratman88} are capable of \textit{simple} adaptations to their
behaviors, implicitly encoded in their \textit{plan library} (a collection of
pre-defined \textit{hierarchical plans} indexed by goals and representing the
standard operations of the domain).
%
This adaptation is due to the fact that \textit{(i)} execution relies entirely
on \textit{context sensitive subgoal expansion}, and therefore, plan choices at
each level of abstraction  are made wrt to the \textit{current}
situation; and  \textit{(ii)} if a plan happens to fail, often because the
environment has changed unexpectedly, agents ``backtrack'' and choose a
different plan-strategy.

However, BDI-style agents are generally unable to go beyond such level of
adaptation. Both the actual behaviours (the plans) and the situations
for which they are appropriate are fixed at deployment time. It is
possible to use meta-level reasoning regarding selections of plans for
particular situations, which may provide substantial flexibility, but
there is no principled learning. In this paper we explore the details
of how effective plan selection can be learnt, based on experience of
what works in which world states.  This mechanism can then be applied
to either refine existing context conditions for plan selection, or
even to learn the context conditions from scratch as we have done in
our experimental work.

% In earlier work \cite{IJATS} we have detailed why we expect that
% there could be problems if the BDI hierarchical structure is not
% taken into account.

%Stephane changed to 'comply with' anonymity 
%I reworked to match own writing style.
In \cite{APSS08} the authors detail why it may be problematic for
learning if the BDI hierarchical structure is not taken into account.
They showed why it can be problematic to assume a mistake at a
higher level in the hierarchy, when a poor outcome may have been
related to a mistake in selection further down.  Their
empirical work suggested that perhaps the advantages of an aggressive
approach, where all failures were considered meaningful, outweighed
the advantages of careful consideration.

In this work, we have used a more sophisticated algorithm to determine
when a decision should be considered sufficiently informed to record a
mistake if it fails to produce a positive outcome. We have 
analysed the kinds of structures which will benefit from careful
consideration vs aggressive learning and demonstrate empirically the
relative advantages of each. 
%
One important finding is that when plans that are unlikely to succeed
(based on current information) are excluded from consideration, the
aggressive approach may never learn, whereas the careful consideration
is much more robust.
% 
In order to address this limitation we have developed an approach which
modifies our confidence in what we have learnt, depending on how much
we have explored the space. We show that this approach, which we call
{\em coverage}\footnote{We use the term coverage because it takes
  account of what portion of the goal plan tree has been {\em covered}
  in our exploration so far.} allows the agressive
approach to be successful, even in those cases where previously it was
unable to learn. Coverage assessment, although orthogonal to the
aggressive vs careful consideration dilemma, does in many ways achieve
the aims of the careful consideration approach. We show experimental
results and conclude that if there is no specialised information about
the structure of the domain, then an agressive approach with coverage
assessment is likely to be most useful for learning context conditions.


The issue of combining learning and deliberative approaches for decision
making of autonomous systems has not been widely addressed. Learning
has been used to learn independent low level skills (for example learn
how to pass a ball for robots in RoboCup soccer (e.g., \cite{Riedmiller01}), which are
then used by the deliberative decision making center of the agent.  In
our work, what is learnt from experience is actually used by the
deliberative component of the BDI agents. Initially, instead of having
a blind exploration strategy 
(as is the case when agents are learning with no initial knowledge), 
% (as is the case when agents are learning from experience and have no initial knowledge), 
our agent has an optimistic strategy: it trusts the existing BDI hierarchy, in other
words, our agent uses existing domain knowledge and keeps on improving
it.

We believe this approach, which takes a BDI program and 
''tune it'' using learning, can be very powerful. While substantial
knowledge is encoded at design and implementation of the system, there
are often additional \textit{nuances} which can be learnt over
time. Particularly in some environments, it may be very difficult to
predict in advance, exactly how certain aspects of the world affect
goal achievability. Our approach allows for encoding what we know,
and learning the rest.



The rest of the paper is organized as follows. In the next section, we
provide a quick overview of typical BDI-style agents.
%
We then discuss modifications to the BDI framework to accommodate learning capabilities by
extending context conditions and the agent plan-selection function. 
%
After that, we describe two approaches to learning the context
condition of plans, aggressive and careful consideration.
%
We report on the empirical results obtained and analyse situations in
which each is advantageous or problematic.
%
We then develop the notion of coverage assessment to influence plan
selection and show that this addresses many of the issues that can
otherwise arise when using the aggressive learning approach.
%
We conclude that care must be taken to
achieve robust and successful learning in a typical BDI hierarchical
program, and that in the absence of specialised information about
program structure, an aggressive learning approach with coverage
assessment during plan selection, is most likely to perform best.
%
%We conclude with a brief summary and discussion.

% The rest of the paper is organized as follows. In the next section, we
% provide an overview of the relevant aspects of typical BDI-style
% agents.
% %
% We then discuss modifications to the BDI framework to include decision
% trees as part of the representation of the context condition, and a
% probabilistic selection which ensures both exploration and
% exploitation. Following this we describe our approach and algorithm,
% which allows to vary the level of care we take before using failure
% data for learning. This is followed by empirical verification and
% discussion of results to illustrate the care that must be taken to
% achieve robust and successful learning in a typical BDI hierarchical
% program. We conclude with a brief summary and discussion.






