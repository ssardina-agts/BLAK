% %%%%%%%%%%%%%%%%%%%%%%%%%%%%%%%%%%%%%%%%%%%%%%%%%%%
\section{Introduction}\label{sec:intro}
% %%%%%%%%%%%%%%%%%%%%%%%%%%%%%%%%%%%%%%%%%%%%%%%%%%%

In this paper, we are concerned with one of the key aspects of the BDI
agent-oriented programming paradigm, namely, that of \emph{intelligent plan
selection} \cite{Pollack92-IRMA,Georgeff89-PRS}.
% %
Specifically, we explore the details of how effective plan selection can be
learnt based on ongoing experience.


There are a plethora of agent programming languages and development platforms in
the BDI tradition, such as  \PRS\ \cite{Georgeff89-PRS}, \JACK~\cite{Busetta99jack},
\TAPL~\cite{Hindriks99:Agent} and \DAPL~\cite{Dastani:JAAMAS08-2APL},
\JASON~\cite{jasonbook}, and SRI's \SPARK~\cite{MorelyM:AAMAS04-SPARK}, among
others. %%
% %
Generally speaking, these systems enable \emph{abstract plans} written by
programmers to be combined and used in real-time, in a way that is both flexible
and robust. Concretely, a BDI agent is built around a
\textit{plan library}, a collection of pre-defined \textit{hierarchical plans}
indexed by goals and representing the standard operational procedures of the
domain (e.g., landing a plane).
% %
The so-called \emph{context condition} attached to each plan states the
conditions under which the plan is a sensible strategy to address the
corresponding goal in a given situation (e.g., it is not raining). The execution
of a BDI system relies then entirely on \textit{context sensitive subgoal
expansion}, allowing agents to ``act as they go'' by making \emph{plan
choices} at each level of abstraction with respect to the current situation.



The fact that both the actual behaviours (the plans) and the situations for which
they are appropriate (their context conditions) are fixed at design time has
important implications for the whole programming approach.
% %
First, it is often difficult or impossible for the programmer to craft the
\emph{exact} conditions under which a plan would succeed. Second, once deployed,
the plan selection mechanism is fixed and may not adapt to potential variations of different environments.
% %
Finally, since plan execution often involves interaction with a \emph{partially
observable} external world, it is desirable to measure success in terms of probabilities rather than boolean values.


%Following~\cite{Airiau:IJAT09}, we propose a BDI programming framework where plans'
%context conditions can be learnt over time, by exploring the plan library and
%suitably incorporating the outcomes into plans' context conditions.
%% %
%To that end, we first extend the standard BDI programming architecture by
%modeling context conditions of plans with \emph{decision trees}
%\cite{Mitchell97:ML} and developing a \emph{probabilistic plan selection scheme}
%compatible with the new representation (Section~\ref{sec:framework}). %%
%% %
%We then  develop and empirically investigate an aggressive approach (called \CL) and a
%conservative one (called \BUL) to learning such context decision trees, and
%showed that they provide different advantages depending on the implicit structure
%of the plan library (Sections~\ref{sec:context_learning} and
%\ref{sec:experiments}). 
The authors have been exploring the nuances in learning within the
hierarchical structure of a BDI program \cite{Airiau:IJAT09} where
it can be problematic to assume a mistake at a
higher level in the hierarchy, when a poor outcome may have been
related to a mistake in selection further down.  In this paper we show
how the scheme which does not take account of this fact, can at times
lead to a complete inability to learn.
We outline two approaches which we have described previously: a conservative approach which takes account
of the structure, considering failures only when decisions made during the execution
are deemed sufficiently ``informed.'',  and an aggressive approach which
ignores the structure and had initially seemed preferable. 
We then describe a new approach which instead of being conservative in
which training examples are used, includes a \emph{confidence}
measure based on how much the agent has explored the space of possible
executions of a given plan (Section~\ref{sec:coverage}). The more this space has
been ``covered'' by previous executions, the more the agent ``trusts'' the
estimation of success provided by the plan's decision tree. This
approach to selection, when combined with the aggressive approach to
training examples achieves a flexible and simple
compromise between the previous two approaches.

Our approach is fully compatible with the usual approach to plan
selection using programmed formula based context conditions. In real
applications we would in fact expect the learning to ``refine''
initially provided selection conditions. 
For simplicity, however, context conditions are learnt from scratch in our
experimental work.

% The rest of the paper is organized as follows. In the next section, we
% provide a quick overview of typical BDI-style agents.
% %
% We then discuss modifications to the BDI framework to accommodate learning capabilities by
% extending context conditions and the agent plan-selection function. 
% %
% After that, we describe two approaches to learning the context
% condition of plans, aggressive and careful consideration.
% %
% We report on the empirical results obtained and analyse situations in
% which each is advantageous or problematic.
% %
% We then develop the notion of coverage assessment to influence plan
% selection and show that this addresses many of the issues that can
% otherwise arise when using the aggressive learning approach.
% %
% We conclude that care must be taken to
% achieve robust and successful learning in a typical BDI hierarchical
% program, and that in the absence of specialised information about
% program structure, an aggressive learning approach with coverage
% assessment during plan selection, is most likely to perform best.
%
%We conclude with a brief summary and discussion.
