Belief, Desire, and Intentions (BDI) agent-oriented programming is a popular 
paradigm for developing complex applications with (soft) real-time reasoning
and control requirements. BDI systems rely on \emph{plan libraries} to achieve
\emph{goals}, and on \emph{context sensitive} subgoal selection and
expansion. 
%
While context conditions do provide the ability for intelligent plan
selection, they are fixed at design time and do not allow agents to
adapt to new environments.
%
We extend the standard BDI programming framework to accommodate a more
flexible and dynamic representation for context conditions and a
(probabilistic) plan selection function that balances both exploration
and exploitation of plans. 
%
We then develop and empirically investigate an aggressive and
a conservative approach to learning context conditions. We demonstrate
that a more conservative
approach that takes account of failure only when it believes
sufficiently informed decisions were taken is more robust.
While each approach has a small advantage in some situations, an
aggressive approach can incur high penalties.

