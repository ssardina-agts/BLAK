%%%%%%%%%%%%%%%%%%%%%%%%%%%%%%%%%%%%%%%%%%%%%%%%%%%%
\section{The Learning Framework}\label{sec:framework}
%%%%%%%%%%%%%%%%%%%%%%%%%%%%%%%%%%%%%%%%%%%%%%%%%%%%

The learning task of interest is as follows: determine whether a plan is applicable in the
current world state given past execution data, as well as conditions programmed by the developer.
We use decision trees ``attached'' to each plan.

Decision trees~\cite{Mitchell97:ML} are a natural classification
mechansim to choose for our purposes as they can deal with noise
(potentially created by an environment with some non-determinism, or
by other factors), and they are able to support disjunctive
hypotheses. Also, they are readily convertible to rules, which are the
standard representation for context conditions.  Data is then
collected regarding the world state at time of selection of the plan,
and whether the plan is considered to have succeeded or
failed. Success data is always recorded, whereas the recording of
failure data may be subject to some analysis as described in the
following section. For the real system we intend to use algorithms for
incremental induction of the decision tree, such as those described in
\cite{Swere06:Fast,Utgoff97Decision}, but for the purpose of our analysis we will use
\textsf{J48} from \textsf{weka} \cite{weka99}, a version of \textsf{c4.5} \cite{Mitchell97:ML}, and
recreate the decision tree after each new piece of data is collected.  We also record the number of
both successful and failed executions of the plan in the different leaf nodes. This
information will be used to estimate the quality of the decision tree, as described below. 


% comments by Stephane: I preferred to write that we record something
% in addition to the weka code so as to make clear that we do not
% modify the way the weka code generate the decision tree (true
% though, we needed to modify some code in weka). Once the tree is
% generated, we use it in a special way.

% We have modified ??? to enable us to extract, for a particular world
% state, the number of both successful and failed executions of the
% plan, as opposed to simply a recommendation that it will succeed or
% fail. This is useful for the calculations regarding when we have
% sufficient information, as described in the following section.

A decision tree will thus give us information for each relevant 
plan, in a particular world state, regarding how likely it
% (or relevant?) plan, in a particular world state, regarding how likely it
is to succeed/fail, based on the agent's experience so far. Given this
information the agent must select which plan to try. There are various
mechanisms for doing this in a standard BDI system, including plan
precedence and meta-level reasoning, 
%stephane adds: I felt a need to justify why adding another mechanism is needed. Feel free to change/remove/improve :-)
but all these mechanisms are pre-programmed and do not take into
account the experience of the agent.
%end add 
However in the context of learning, we must consider the standard
dilemma of exploitation vs exploration.  We have taken a fairly simple
approach to this which appears to work well. We
% simply normalise the information regarding the relative success of
% each plan, and then
select a plan using a probabilistic function, which selects a plan
with a probability proportional to its relative success.

In the following work we make a number of assumptions to enable us to
explore the basic issues we are concerned with, and to be able to more
easily compare two extreme approaches. These assumptions are:
\begin{tightenumerate}
\item
Actions in the environment may fail with some probability, even in
world states where they ''should'' succeed. 
This captures the non-deterministic nature of most real
world environments. 
\item
No changes happen in the environment, during execution, other than
those made by the agent. (Although this may appear too
limiting, the previous assumption of non-determinism mitigates against
the over-simplification for many cases, by treating other changes as
non-deterministic failures.)
\item
The agent is only executing a single intention (i.e., pursuing a single
goal). 
%
Without this assumption it would be necessary for the agent to
reason about potentially negative interactions between concurrent goals. While such reasoning is
possible
\cite{Thangarajah02}, it complicates the situation for this work.
% Without this assumption it would be necessary for the agent to
% reason about potentially damaging changes it may have itself made, in
% pursuit of a different goal. While such reasoning is possible
% \cite{Thangarajah02}, it complicates the situation for this work.

\item
There is no automated failure handling, whereby the BDI system retries
a goal with an alternative plan, if the selected plan fails. This will
certainly be integrated into the system for real use, but again
complicates the understanding of the basic mechanisms. In fact,
its integration may help achieve faster learning.
%  for real applications. 
\end{tightenumerate}



