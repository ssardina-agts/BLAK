%!TEX root = ../ijcai11storage.tex

%The Belief-Desire-Intentions (BDI) agent programming paradigm is a
%successful and popular approach to developing complex systems that
%will behave robustly in situations where the environment is
%interacting with the agent system. 
In this paper we present a
framework for integrating learning into the popular and robust BDI
agent programming paradigm. We summarise
previous work which has integrated decision trees into the context
condition used for plan selection, and then develop a confidence
measure which allows the agent to adjust its reliance on the
decision tree dynamically, facilitating both initial learning and
re-learning. We evaluate this with an example of an embedded controller
for energy management.

%% Abstract submissions for AAMAS 2011 are limited to 400 words maximum
%We propose enhancements to a framework that integrates
%learning capabilities to improve plan selection in the successful and popular
%Belief-Desire-Intentions agent programming paradigm.
%
%In learning which plan to select, a 
%crucial issue in the online setting is how much to trust what
%has been learnt so far (and therefore exploit it) versus how much to
%explore to further improve the learning.
%
%In this paper we construct a confidence measure based on a previously 
%used notion of stability in the outcomes observed for a particular plan, 
%combined with a consideration of the extent to which new worlds are being 
%witnessed by the plan.
%
%This new measure dynamically adjusts based on agent performance,
%allowing in principle, infinitely many learning phases. Additionally,
%it scales up irrespective of the complexity of the goal-plan hierarchy
%implicit in the agent's plan library. 
%
%We demonstrate the utility of our approach with results obtained in a
%practical energy storage domain.

%%% 3. Lin's Take
%In this work we propose an important modification to a framework for
%agent-oriented programming that integrates plan selection
%learning capabilities to the successful and popular
%Belief-Desire-Intentions programming paradigm.
%In learning which plan to select, a 
%crucial issue in the online learning setting is how much to trust what
%has been learnt so far (and therefore exploit it) versus how much to
%explore further and increase the information on which the learning is
%based. 
%Previous work has employed a confidence measure that is based on an
%estimate of how much of the space of options has been explored. 
%%A problem with this approach however is that the confidence measure does
%%not adjust to a possibly changing situation/environment. 
%In this paper we take a notion of stability in the outcomes observed
%for a particular plan, which was previously used for filtering
%training data, and adapt it to form the basis of a new and more robust
%confidence measure. We add to this a consideration of to what extent
%we are still seeing new world situations.  Unlike earlier approaches
%this new confidence measure adjusts dynamically and thus allows, in
%principle, infinitely many learning and re-learning phases as things
%change. In addition it scales up without problem, regardless of the
%complexity of the goal-plan hierarchy.
%We demonstrate the utility of our approach with results obtained in a
%practical energy storage domain. 



%%% 2. Sebastian's Take
%We propose a framework for agent-oriented programming that integrates learning capabilities to the successful and popular Belief-Desire-Intentions programming paradigm for dealing with the crucial task of intelligent plan selection.
%%
%In contrast with previous proposals, the learning framework developed here is able to scale up irrespective of the complexity of the goal-plan hierarchy implicit in the agent's plan library and to adapt to changes in the dynamics of the environment.
%%
%Technically, we propose a new and simple way of determining the ``confidence'' in the ongoing learning process of the agent that adjusts dynamically based on agent performance, thus allowing, in principle, infinitely many learning phases. when used in the exploration heuristic. 
%
%We demonstrate the utility of this agent-oriented learning framework with results obtained in a practical energy storage domain.


%%% 1. Dhirendra's Take
% The popular  (BDI)  has been applied to a range of real-world applications. Recent work has proposed the principled integration of a learning capability to the BDI architecture in order to extend applicability to domains where adaptability is important. In particular, the question being addressed is that of {\em plan choice}, or learning which plans work best in which situations. In this paper we report new progress in this direction.
% %
% 
% Firstly, we contribute to the important issue of determining confidence in the ongoing learning of the agent. We propose a new measure that is truly scalable irrespective of the size of the BDI goal-plan hierarchy. Additionally, this measure dynamically adjusts based on agent performance, allowing in principle, infinitely many learning phases when used in the exploration heuristic. 
% %
% 
% The task is to program a controller for a modular battery system
% while ensuring adaptability to certain future performance-impacting
% situations such as changes in battery chemistry and module
% malfunctions. This learning scenario is typical of many real-world
% applications, but highlights a shift from the usual setting: here
% the task is not to learn the initial solution set, but to program
% the initial set and then use learning to adapt to changes to this
% set over time. A key consideration then is the programming of the
% initial {\em filter} that must allow for the ideal solution set as
% well as any future sets that are to be learnt. This is achieved in
% our BDI agent by programming each plan's {\em context condition}
% (the runtime condition that decides when the plan is applicable) in
% such a way as to capture all foreseeable solution sets. Learning is
% then used to refine these conditions over time in response to
% different environmental changes. 
% 
