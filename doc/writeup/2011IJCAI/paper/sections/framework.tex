%!TEX root = ../ijcai11storage.tex
%\newpage 
% %%%%%%%%%%%%%%%%%%%%%%%%%%%%%%%%%%%%%%%%%%%%%%%%%%%
\section{The Integrated BDI Learning Framework}\label{sec:framework}
% %%%%%%%%%%%%%%%%%%%%%%%%%%%%%%%%%%%%%%%%%%%%%%%%%%%

... To that end, it was proposed to generalize the account for plans' context conditions to decision trees~\cite{Mitchell97:ML} that can be learnt over time~\cite{airiau09:enhancing,singh10:extending,singh10:learning}. The idea is simple: \emph{the decision tree of an agent plan provides a judgement as to whether the plan is likely to succeed or fail for the given situation.}
%%
By suitably \emph{learning} the structure of and adequately \emph{using} such decision trees, the agent is able to improve its performance over time 
%and release 
, lessening the burden on
the domain modeller to encode ``perfect'' plan preconditions. Note that the classical boolean context conditions provided by the designer could (and generally will) still be used as initial necessary but possibly insufficient requirements for each plan that will be further \emph{refined} over time in the course of trying plans in various world states.


Under the new BDI learning framework, two mechanisms become crucial. First, of course, a principled approach to learning such decision trees based on execution experiences is needed. Second, an adequate plan selection scheme compatible with the new type of plans' preconditions is required.
%%
To select plans based on information in the decision trees, the work reported in \cite{singh10:extending,singh10:learning} used a probabilistic method that chooses a plan based on its believed likelihood of success in the given situation. This approach provides a balance between exploitation (by choosing plans with relatively higher success expectations more often), and exploration (by sometimes choosing plans with lower success expectation to get better confidence in their believed applicability by trying them in more situations). This balance is important because ongoing learning influences future plan selection, and subsequently whether a good solution is found.

When it comes to the learning process, the training set for a given plan's decision tree contains samples of the form $[w, o]$, where $w$ is the world state---a vector of discrete attributes---in which the plan was executed and $o$ is the execution outcome, namely, success or failure. Initially, the training set is empty and grows as the agent tries the plan in various world states and records each execution result. 
%%
Since the decision tree inductive bias is a preference for smaller trees, one expects that the learnt decision tree consists of only those world attributes that are relevant to the corresponding plan's (real) context condition.
%%

Due to the hierarchical nature of the goal-plan structure being executed by the agent (c.f. Figure~\ref{fig:confidence}), it is possible that the failure of a plan (e.g., $P$) is only due to a poor plan selection lower in the hierarchy (e.g., $P_l$ is selected for goal $G_4$).
%%
To deal with this issue, the work in \cite{airiau09:enhancing} uses a plan ``stability" measure to take failures into account only when the agent is sufficiently sure that the failure was not due to poor sub-plan choices. 
%
To further understand this notion consider the case where plan selection results in the failed execution trace $\lambda = G[P:w] \cdot G_2[P_f:w_2] \cdot G_5[P_n:w_5]$.\footnote{Note that trace $\lambda$ assumes the successful resolution of subgoal $G_1$ and the resulting world state $w'$, as described by $\lambda_1$ previously.} What should we make of this failure from a learning perspective? Should the negative outcome be recorded for training our decision tree at non-leaf nodes $P_f$ and $P$? The concern stems from the fact that these non-leaf plans failed not because they were a bad choice for world $w$ but because a bad choice ($P_n$) was made further down in the hierarchy. To resolve this issue, the stability filter is used in \cite{airiau09:enhancing} to record failures only for those plans whose outcomes are considered to be stable, or ``well-informed.'' 
%

Another approach reported previously is to adjust the plan selection probability based on some measure of ``confidence'' in the decision tree~\cite{singh10:extending,singh10:learning} which considers the reliability of a plan's decision tree to be proportional to the number of sub-plan choices (or paths below the plan in the goal-plan hierarchy) that have been explored already: the more choices that have been explored, the greater the confidence in the resulting decision tree. 


% %%%%%%%%%%%%%%%%%%%%%%%%%%%%%%%%%%%%%%%%%%%%%%%%%%%
\subsection{A Dynamic Confidence Measure}\label{sec:confidence}
% %%%%%%%%%%%%%%%%%%%%%%%%%%%%%%%%%%%%%%%%%%%%%%%%%%%

In this section we describe a dynamic confidence measure that may be used to guide exploration when learning plan selection using the framework described in Section \ref{sec:framework}. Conceptually, the value of the confidence measure relates to the degree of trust that the agent has in its current understanding of the world (from a learning perspective). Technically, recall that the confidence metric informs the agent about how much it should trust the outcome estimate provided by its current decision tree.

Our new confidence measure improves upon previously used measures in two important ways. 
%
Firstly, it caters to changing dynamics of the environment that often results in prior learning becoming less effective. The stability-based \cite{airiau09:enhancing} and coverage-based \cite{singh10:extending,singh10:learning} measures that have been previously proposed do not support the requirement for adaptability to such changes. Moreover, the new measure proposed here subsumes the functionality of the former methods, as it behaves monotonically in environments where the dynamics are fixed. As such, it offers a direct replacement for the previous approaches. 
%
Secondly, the new measure does not rely on estimates of the number of choices in the goal-plan hierarchy as is the case in \cite{singh10:extending,singh10:learning}, and scales to any general goal-plan hierarchy irrespective of its complexity.


To recap the definition of stability from \cite{singh10:learning}:

\begin{quote}
\emph{``A failed plan $P$ is considered to be stable for a particular world state $w$ if the rate of success of $P$ in $w$ is changing below a certain threshold.''}
\end{quote} 

\newcommand{\ds}{\zeta}
\newcommand{\app}{\mathname{app}}
\newcommand{\stable}{\mathname{stable}}

Our aim is to use this notion to judge how ``stable'' or well-informed the decisions the agent has made within a particular execution trace were. This is particularly meaningful for \emph{failed} execution traces: low stability suggests that we were not well-informed and more exploration is needed before assuming that no solution is possible (for the trace's top goal in question).
%%
To capture this, we define the \emph{degree of stability} of a (failed) execution trace $\lambda$, denoted $\ds(\lambda)$ as the ratio of stable plans to total applicable plans in the active execution trace below the top-level goal event in $\lambda$. Formally, when $\lambda= G_1[P_1:w_1] \cdots G_n[P_n:w_n]$ we define 
%%
\[
\ds(\lambda) = 
	\frac{ 
			\card{ \bigcup_{i=1}^n \set{P \mid P \in \Delta_{\app}(G_i,w_i),\, \stable(P,w_i)} } 
		}
		{
			\card{	\bigcup_{i=1}^n \Delta_{\app}(G_i,w_i) } 
		},
\]

\noindent
where  $\Delta_{\app}(G_i,w_i)$ denotes the set of all applicable plans (i.e., whose boolean context conditions hold true) in world state $w_i$ for goal event $G_i$, and $\stable(P,w_i)$ holds true if plan $P$ is deemed stable at world state $w_i$, as defined in~\cite{singh10:learning}.

For instance, take the failed execution trace $\lambda = G[P:w] \cdot G_2[P_f:w_2] \cdot G_5[P_n:w_5]$ from before and suppose further that the applicable plans are $\Delta_{\app}(G,w) = \{P\}$, $\Delta_{\app}(G_2,w_2) = \{P_d,P_f\}$, and $\Delta_{\app}(G_5,w_5) = \{P_m,P_n,P_o\}$. Further assume that $P_d$ and $P_n$ are the only plans deemed stable (in worlds $w_2$ and $w_5$ respectively). 
%%
Then the degree of stability for the whole trace is $\ds(\lambda)= 2/6$.
%%
Similarly, for the two subtraces $\lambda'= G_2[P_f:w_2] \cdot G_5[P_n:w_5]$ and $\lambda'' =G_5[P_n:w_5]$ of $\lambda$, we get $\ds(\lambda') = 2/5$ and $\ds(\lambda'') = 1/3$.



\newcommand{\StablePlan}{\mathname{StablePlan}}
\newcommand{\SetDegreeStability}{\mathname{RecordDegreeStability}}
\newcommand{\UpdateDegreeStability}{\mathname{RecordDegreeStabilityInTrace}}

The idea is that every time the agent reaches a failed execution trace, the stability degree of each subtrace is stored in the plan that produced that subtrace.
%%
So, for our example, for plan $P$ we store degree $\ds(\lambda')$ whereas for plan $P_f$ we record degree $\ds(\lambda'')$. Leaf plan nodes, like $P_n$, make no choices so their degree is simply $1$.
%%
Intuitively, by doing this, we record against each plan in the (failed) trace, an estimate of how informed the current (active) choices made for the plan were.  
%%
Algorithm~\ref{alg:degree} describes how this (hierarchical) recording happens given an active execution trace $\lambda$. Observe how the stability measure is recorded against each plan in the trace: $\SetDegreeStability(P, w, d)$ records degree $d$ for plan $P$ in world state $w$.

\begin{algorithm}[h]
\KwData{$\lambda=G_1[P_1:w_1] \cdot \ldots \cdot G_n[P_n:w_n]$, with $n \geq 1$.}
\KwResult{Records degree of stability for plans in $\lambda$.}
\If{$(n > 1)$}{
	$\lambda'=G_2[P_2:w_2] \cdot \ldots \cdot G_n[P_n:w_n]$\;
	$d = \ds(\lambda')$\;  
	$\SetDegreeStability(P_1, w_1, d)$\;
	$\UpdateDegreeStability(\lambda')$\;
}
\Else{$\SetDegreeStability(P_1, w_1, 1)$\;}
\caption{$\UpdateDegreeStability(\lambda)$}
\label{alg:degree}
\end{algorithm}


As a plan execution produces new failed experiences, the calculated degree of stability is appended against it each time. When a plan finally succeeds, we take an optimistic view and record $1$ (i.e., full stability) against it. This, together with the fact that all plans do eventually become stable, means that $\ds(\lambda)$ is guaranteed to converge to $1$. 


To aggregate the different stability recordings for a plan over time, we use the \emph{average degree of stability} over the last $n \geq 1$ executions of plan $P$ in $w$, denoted $\C_s(P,w,n)$. 
%%
This provides us with a measure of confidence in the decision tree for plan $P$ in state $w$. Intuitively, $\C_s(P,w,n)$ tells us how ``informed" the decisions taken when performing $P$ in $w$ were over the $n$ most recent executions.
%%
Notice that if the environment dynamics are constant, this measure monotonically increases from $0$, as plans below $P$ start to become stable (or succeed); it reaches $1$ when all tried plans below $P$ in the last $n$ executions are considered stable. This is what one might expect in the typical learning setting. However, if the environment dynamics were to change and plans start to fail or become unstable, then the measure behaves non-monotonically and adjusts confidence accordingly.

The stability-based confidence measure $\C_s(\cdot,\cdot,\cdot)$ would make a useful heuristic for exploration (i.e., plan selection) in its own right: when the confidence is at its lowest the agent does maximum exploration, and when it is at its highest, the agent fully utilises the decision trees. 
%%
A problem with this approach, though, is that such measure only covers the space of \emph{known} worlds. This means that whenever a new world is witnessed, this stability-based confidence will be zero, meaning that the agent will choose randomly. This is hardly beneficial since what we would really like is to use the learnt generalisations to classify (i.e. predict the outcome in) this new world rather than being agnostic about it. 
%%
What is missing is a complementary metric that contributes to our net confidence but that is independent of $w$.


\newcommand{\neww}{\mathname{NewStates}}
One way to address this limitation is by monitoring the rate at which new worlds are being witnessed by a  plan $P$. During early exploration, it is expected that the majority of worlds that a plan is selected for will be unique, thus yielding a high rate and a low confidence. Over time, as exploration continues, the plan would get selected in all worlds in which it is reachable and the rate of new worlds would approach zero, while our confidence over this period increases to its maximum.  
%%
So, we define our second confidence metric $\C_d(P,n) = |\neww(P,n)|/n$, where $\neww(P,n)$ is the set of world states that have been seen for the first time in the last $n$ executions of $P$.
%%
Clearly, $\C_d$ is guaranteed to converge to $1$ as long as all worlds where the plan might apply are eventually witnessed.

In summary, we have defined two confidence metrics over two orthogonal dimensions. Stability-based confidence $\C_s(P,w,n)$ is meant to capture how well-informed the last $n$ executions of plan $P$ in world $w$ were, whereas world-based confidence $\C_d(P,n)$ is meant to capture how well-known the worlds in the last $n$ executions of plan $P$ were, compared with what we had experienced before.


We now have all the technical machinery to define our final (aggregated) confidence measure $\C$ that takes into account both the above metrics. Specifically, the overall confidence in the decision tree of plan $P$ in world $w$ relative to the last $n$ experiences is defined as follows:
%%
\[
	\C(P,w,n) = \alpha\C_s(P,w,n) + (1-\alpha)\C_d(P,n),
\label{eqn:confidence}
\]

\noindent
where $\alpha$ is a weighting factor used to set a preference bias between the two component metrics.



Finally, we show how this confidence measure is to be used within the plan selection mechanism of a BDI agent. Remember that for a given goal-event that needs to be resolved, a BDI agent may have several applicable plans from which one ought to be selected for execution. The BDI learning framework described in Section~\ref{sec:framework} will chose probabilistically among these options in a way proportional to some given weight per plan---the more weight a plan is assigned, the higher the chances of it being chosen. 
%%
Following~\cite{singh10:extending,singh10:learning}, we define this selection weight for plan $P$ in world $w$ relative its last $n$ executions as follows: 
%%
\[
	\Omega(P,w,n) = 0.5 + \left[  \C(P,w,n) \times  \left( \P(P,w) - 0.5 \right)  \right],
\label{eqn:omega}   
\]

\noindent 
where $\P(P,w)$ is the probability of success of plan $P$ in world $w$ as given by $P$'s decision tree. 
%
Indeed, this weight definition is basically that of~\cite{singh10:extending,singh10:learning}, except for the use of our new confidence metric $\C(\cdot,\cdot,\cdot)$ as defined above. The idea is to combine the likelihood of success of plan $P$ in world $w$ (i.e. the term $\P(P,w)$) with a confidence bias (in this case $\C(\cdot,\cdot,\cdot) \in [0.0:1.0]$) to determine a final plan selection weight $\Omega(\cdot,\cdot,\cdot)$. 
When the confidence is maximum i.e. $1.0$, then $\Omega(\cdot,\cdot,\cdot) = \P(\cdot,\cdot)$, and the final weight equals the likelihood reported by the decision tree; when the confidence is zero, then $\Omega(\cdot,\cdot,\cdot)=0.5$, and the decision tree has no bearing on the final weight (a default weight of $0.5$ is used instead).


This mechanism for probabilistic plan selection using dynamic confidence, together with the decision tree integration explained in Section~\ref{sec:framework} completely describes our BDI learning framework. Given this, in the subsequent section we cover the use of this framework in the design of a complete BDI system for an energy storage scenario. The application is a fully functional implementation of a modular battery controller from initial specification. More importantly, by incorporating the learning framework, we shall demonstrate that the program is able to handle (certain) foreseeable changes in operational dynamics for the battery system once deployed.

