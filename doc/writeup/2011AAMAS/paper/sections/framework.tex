%!TEX root = ../aamas11storage.tex
%\newpage 
% %%%%%%%%%%%%%%%%%%%%%%%%%%%%%%%%%%%%%%%%%%%%%%%%%%%
\section{The BDI Learning Framework}\label{sec:framework}
% %%%%%%%%%%%%%%%%%%%%%%%%%%%%%%%%%%%%%%%%%%%%%%%%%%%

We begin by briefly reviewing the basic agent programming framework that will be used throughout the paper~\cite{airiau09:enhancing,singh10:extending,singh10:learning}, which is a seamless integration of standard Belief-Desire-Intention (BDI) agent-oriented programming~\cite{Rao96:AgentSpeak,WooldridgeBook} with decision tree learning~\cite{Mitchell97:ML}. 

Generally speaking, BDI agent-oriented programming languages are built around an explicit representation of propositional attitudes (e.g., beliefs, desires, intentions, etc.). A BDI architecture addresses how these components are represented, updated, and processed to determine the agent's actions.
%%
There are a plethora of agent programming languages and development platforms in the BDI tradition, including
%such as \PRS\ \cite{Georgeff89-PRS},
\JACK~\cite{BusettaRHL:AL99-JACK}, 
\JADEX~\cite{Pokahr:EXP03-JADEX}, and
%\TAPL~\cite{Hindriks99:Agent} and
%\DAPL~\cite{Dastani:JAAMAS08-2APL}, 
\JASON~\cite{jasonbook}
%, and SRI's \SPARK~\cite{MorelyM:AAMAS04-SPARK}, 
among others. 
%%
Specifically, a BDI intelligent agent systematically chooses and executes \emph{plans} (i.e., operational procedures) to achieve or realize its goals, called \emph{events}.
%%
Such plans are extracted from the so-called agent's \emph{plan library}, which encode the ``know-how'' information of the domain the agent operates on.
%%
For instance, the plan library of an unmanned air vehicle (UAV) agent controller may include several plans to address the event-goal of landing the aircraft. Each plan is associated with a \emph{context condition} stating under which belief conditions the plan is a sensible strategy for addressing the goal in question. Whereas some plans for landing the aircraft may only be suitable under normal weather conditions, other plans may only be used under emergency operations.
%%%
Besides the actual execution of domain actions (e.g., lifting the flaps), a plan may require the resolution of (intermediate) sub-goal events (e.g., obtain landing permission from air control tower). As a result, the execution of a BDI system can be seen as a \textit{context sensitive subgoal expansion}, allowing agents to ``act as they go'' by making \emph{plan choices} at each level of abstraction with respect to the current situation. The use of plans' context (pre)conditions to make choices as late as possible, together with the built-in goal-failure mechanisms, ensures that the system is responsive to changes in the environment. 




It is not hard to see that, by grouping together plans responding to the same event type, the agent's plan library can be seen as
a set of goal-plan tree templates (e.g., Figure~\ref{fig:confidence}): a goal-event node (e.g., goal $G_1$) has children representing the alternative \emph{relevant} plans for achieving it (e.g., $P_a,P_b$ and $P_c$); and a plan node (e.g., $P_f$), in turn, has children nodes representing the subgoals (including primitive actions) of the plan (e.g., $G_4$ and $G_5$). These structures, can be seen as AND/OR trees: for a plan to succeed all the subgoals and actions of the plan must be successful (AND); for a subgoal to succeed one of the plans to achieve it must succeed (OR). Leaf plans are meant to directly interact with the environment and so, in a given world state, they can either succeed or fail when executed; this is marked accordingly in the figure for some particular world (of course such plans may behave differently in other states).


As can be seen, adequate plan selection is critically important in BDI systems. Whereas standard BDI platforms leverage domain expertise by means of the \emph{fixed} logical context conditions of plans, in this work, we are interested in exploring how a situated agent may \emph{learn} or \emph{improve} its plan selection mechanism based on experience, in order to better realize its goals.
%%
To that end, it was proposed to generalize the account for plans' context conditions to decision trees~\cite{Mitchell97:ML} that can be learnt over time~\cite{airiau09:enhancing,singh10:extending,singh10:learning}. The idea is simple: \emph{the decision tree of an agent plan provides a judgement as to whether the plan is likely to succeed or fail for the given situation.}
%%
By suitably \emph{learning} the structure of, and adequately \emph{using} such decision trees, the agent is able to improve its performance over time 
%and release 
, lessening the burden on
the domain modeller to encode ``perfect'' plan preconditions. Note that the classical boolean context conditions provided by the designer could (and generally will) still be used as initial necessary but possibly insufficient requirements for each plan that will be further \emph{refined} over time in the course of trying plans in various world states.


Under the new BDI learning framework, two mechanisms become crucial. First, of course, a principled approach to learning such decision trees based on execution experiences is needed. Second, an adequate plan selection scheme compatible with the new type of plans' preconditions is required.
%%
To select plans based on information in the decision trees, the work reported in \cite{singh10:extending,singh10:learning} used a probabilistic method that chooses a plan based on its believed likelihood of success in the given situation. This approach provides a balance between exploitation (by choosing plans with relatively higher success expectations more often), and exploration (by sometimes choosing plans with lower success expectation to get better confidence in their believed applicability by trying them in more situations). This balance is important because ongoing learning influences future plan selection, and subsequently whether a good solution is found.


\begin{figure}[t]
\begin{center}
%\resizebox{.45\textwidth}{!}{
%!TEX root = ../ijcai11storage.tex
\begin{tikzpicture}[level distance=1.0cm]
\tikzstyle{succ}=[label=below:$\surd$]
\tikzstyle{fail}=[label=below:$\times$]
\tikzstyle{trace}=[solid,style=circle,fill=black!20,above left=0.15cm,scale=.7]

\tikzstyle{planbox}=[draw,minimum height=0.6cm,minimum width=0.65cm]
\tikzstyle{goalbox}=[draw,minimum height=0.6cm,minimum width=0.65cm,rounded corners=0.2cm]
\tikzstyle{planbox2}=[planbox,pattern=north west lines,pattern color=black!30]
\tikzstyle{goalbox2}=[goalbox,pattern=north west lines,pattern color=black!30]
\tikzstyle{goalbox3}=[goalbox,fill=black!20]
\tikzstyle{planbox3}=[planbox,fill=black!20]

	
\tikzstyle{level 2}=[sibling distance=4cm] 
\tikzstyle{level 3}=[sibling distance=1cm] 
\tikzstyle{level 4}=[sibling distance=2.7cm]
\tikzstyle{level 5}=[sibling distance=0.9cm]

\node[goalbox2] {$G$}
child { node[planbox3,solid] {$P$}
		child {node[goalbox2] {$G_{1}$}
			child {node[planbox2] {$P_{A}$}
				child {node[goalbox2] {$G_{3}$}
					child {node[planbox,fail] {$P_{G}$}}
					child {node[planbox,fail] {$P_{H}$}}
					child {node[planbox2,succ] {$P_{I}$}}
				}
			}
			child {node[planbox,fail] {$P_{B}$}}
			child {node[planbox,fail] {$P_{C}$}}
		}
		child {node[goalbox3] {$G_{2}$}
			child {node[planbox,fail] {$P_{D}$}}
			child {node[planbox,fail] {$P_{E}$}}
			child {node[planbox3] {$P_F$}
				child {node[goalbox2] {$G_{4}$}
					child {node[planbox,fail] {$P_{J}$}}
					child {node[planbox2,succ] {$P_{K}$}}
					child {node[planbox,fail] {$P_{L}$}}
				}
				child {node[goalbox3] {$G_{5}$}
					child {node[planbox3,succ] {$P_{M}$}}
					child {node[planbox,fail] {$P_{N}$}}
					child {node[planbox,fail] {$P_{O}$}}
				}
			}
		}
};

\end{tikzpicture}

%}
\end{center}
\caption{An example goal-plan hierarchy.}
\label{fig:confidence}
\end{figure}

When it comes to the learning process, the training set for a given plan's decision tree contains samples of the form $[w, o]$, where $w$ is the world state---a vector of discrete attributes---in which the plan was executed and $o$ is the execution outcome, namely, success or failure. Initially, the training set is empty and grows as the agent tries the plan in various world states and records each execution result. 
%%
Since the decision tree inductive bias is a preference for smaller trees, one expects that the learnt decision tree consists of only those world attributes that are relevant to the corresponding plan's (real) context condition.
%%
Now, due to the hierarchical nature of the plan-goal hierarchy being executed by the agent (c.f. Figure~\ref{fig:confidence}), it is possible that the failure of a plan (e.g., $P$) is only due to a poor plan selection lower in the hierarchy (e.g., $P_l$ is selected for goal $G_4$).
%%
To deal with this issue, the work in \cite{airiau09:enhancing} uses a plan ``stability" measure to take failures into account only when the agent is sufficiently sure that the failure was not due to poor sub-plan choices. Another approach reported is to adjust the plan selection probability based on some measure of ``confidence'' in the decision tree~\cite{singh10:extending,singh10:learning} which considers the reliability of a plan's decision tree to be proportional to the number of sub-plan choices (or paths below the plan in the goal-plan hierarchy) that have been explored already: the more this space has been explored, the greater the confidence in the resulting decision tree. 



\textbf{Shouldn't we talk here about execution traces and even give an example? Also a bit more on stability?}  

