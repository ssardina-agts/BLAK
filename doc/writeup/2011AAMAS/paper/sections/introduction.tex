%!TEX root = ../aamas11storage.tex
% %%%%%%%%%%%%%%%%%%%%%%%%%%%%%%%%%%%%%%%%%%%%%%%%%%%
\section{Introduction}\label{sec:introduction}
% %%%%%%%%%%%%%%%%%%%%%%%%%%%%%%%%%%%%%%%%%%%%%%%%%%%

BDI agents do not traditionally do learning. However, if a deployed agent
(or agent system) is to be able to adapt over time to a changing
situation, then learning becomes important. Our vision is to be able
to deploy an agent system which will be capable of adjusting to
ongoing changes in an environment that were not foreseen and
programmed for at deployment time, in a robust and effective manner.

In this paper we build on the work of \cite{} which provides
mechanisms for a BDI agent to learn the appropriate selection of plans
in a situation. One of the key issues in any learning system is
exploitation vs exploration: i.e. how much to believe and exploit the
current knowledge, vs how much to try things in order to gain new
knowledge. One of the key factors in this balance is how much learning
has already been done, or how much has already been explored. In the
work on which this is based, a confidence factor was used to capture
how much the agent should trust its current understanding, and
therefore explot rather than explore. However, this confidence factor
was based on a static approximation based on the structure of the
space of choices. The greater the percentage of the space that had
been explored, the greater the confidence, and consequently the more
likely the agent is to exploit.  

However, this approach does not work for achieving our vision of an
agent able to adjust to a changing environment, as the confidence
increases monotonically. Because of this monotonic increase in
confidence there is no ability for the agent
to become less confident in its choices if the environment changes,
and its currently learned behaviour becomes less successful.
What we would prefer, and what we present in this paper, is a
mechanism whereby the agent can observe how stable its learned
knowledge is, and based on this stability, have a confidence factor
which affects the balance between exploration and exploitation. We
develop this new confidence measure, based on a notion of stability
also developed in \cite{}.  An additional critical factor in an
intuitive notion of confidence, is how much we are experiencing states
we have seen before, vs how much we are seeing new situations. If we
are still experiencing a substantial number of new situations we
should be less confident that we have learnt well how to operate in
the world. This notion is also incorporated into our new confidence
measure by using a sliding window that checks what percentage of
situations in the window have been seen previously.

In the following section we provide an overview of the basic BDI
learning framework that we are using, as developed by \cite{}. We then
explain in detail our new dynamic measure of confidence, that enables
our agents to continually adjust to a changing world. We show how this
approach is much more suitable for our battery controller
agent. Finally we describe some experiments evaluating the battery
controller agent in different situations, and demonstrate that our
approach does in fact allow the agent to adjust. in a variety of ways,
to an environment where battery behaviour changes. We finish with
conclusions and comments on what has been achieved as well as
limitations requiring future work.

*** need to introduce the example in intro...
