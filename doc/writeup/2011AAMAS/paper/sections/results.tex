%!TEX root = ../ijcai11storage.tex
\begin{figure*}[t]
 \begin{center}
 \subfigure[Goal-plan hierarchy for a $k$-modules battery system.]{
 \label{fig:gptree}
 \resizebox{0.9\columnwidth}{!}{
 %!TEX root = ../aamas11storage.tex
\begin{tikzpicture} [level distance=8.0em]
\tikzstyle{planbox}=[draw,text width=11.0em,rectangle split,rectangle split parts=3]
\tikzstyle{goalbox}=[draw,rounded corners=1.25em,minimum height=3em,minimum width=5em]

	
\tikzstyle{level 1}=[sibling distance=13.0em] 
\tikzstyle{level 2}=[level distance=7.0em] 

\node[goalbox,solid] {$G($r,k,s$)$}
	child {node[planbox] {$SetCharge$ 
			\nodepart{second} $\psi:satisfies(r,k,s,C),$\\$k>0$
			\nodepart{third} $set(k,C)$
		}
		child {node[goalbox] {$G($r,k-1,s'$)$}}
	}
	child {node[planbox] {$SetDischarge$ \nodepart{second}
			\nodepart{second} $\psi:satisfies(r,k,s,D),$\\$k>0$
			\nodepart{third} $set(k,D)$
		}
		child {node[goalbox] {$G($r,k-1,s'$)$}}
	}
	child {node[planbox] {$SetNotUsed$ \nodepart{second}
			\nodepart{second} $\psi:satisfies(r,k,s,N),$\\$k>0$
			\nodepart{third} $set(k,N)$
		}
		child {node[goalbox] {$G($r,k-1,s'$)$}}
	}
	child {node[planbox] {$Execute$ 
			\nodepart{second} $\psi:k==0$
			\nodepart{third} $operate()$ \\$evaluate()$
		}
	}
;

\end{tikzpicture}



 }
 }
 %\qquad
 \subfigure[Capacity deterioration.]{
 	\label{fig:experiment1}
	%!TEX root = ../ijcai11storage.tex
\begin{tikzpicture}

\begin{axis}[
width=0.28\columnwidth,height=3cm,scale only axis,
axis line style={-}, xtick style={-}, ytick style={-},
%xlabel=Episodes,
%ylabel=Success,
every axis y label/.style={at={(-0.12,0.5)},rotate=90,anchor=center}, 
%every axis x label/.style={at={(0.5,-0.15)},anchor=center},
grid=both,grid style={-,style=densely dotted},
legend style={at={(0.5,0.25)},anchor=north west}
] 

\addplot[-] file {./data/storage1b.CF.tikzdata};
%\addlegendentry{Data} 

\end{axis} 
\end{tikzpicture} 

 }
 \hspace{-3mm}
 \subfigure[Partial failure.]{
 	\label{fig:experiment2}
 	%!TEX root = ../ijcai11storage.tex
\begin{tikzpicture}

\begin{axis}[
width=0.5\columnwidth,height=4cm,scale only axis,
axis line style={-}, xtick style={-}, ytick style={-},
%xlabel=Episodes,
%ylabel=Success,
every axis y label/.style={-,at={(-0.12,0.5)},rotate=90,anchor=center}, 
%every axis x label/.style={at={(0.5,-0.15)},anchor=center},
grid=both,grid style={-,style=densely dotted},
legend style={at={(0.5,0.25)},anchor=north west}
] 

\addplot[-] file {./data/storage2b.CF.tikzdata};
%\addlegendentry{Data} 

\end{axis} 
\end{tikzpicture} 

 }
 \hspace{-3mm}
 \subfigure[Complete failure.]{
 	\label{fig:experiment3}
	 %!TEX root = ../aamas11storage.tex
\begin{tikzpicture}

\begin{axis}[
width=0.8\columnwidth,height=4cm,scale only axis,
axis line style={-}, xtick style={-}, ytick style={-},
xlabel=Episodes,
ylabel=Success,
every axis y label/.style={-,at={(-0.12,0.5)},rotate=90,anchor=center}, 
%every axis x label/.style={at={(0.5,-0.15)},anchor=center},
grid=both,grid style={-,style=densely dotted},
legend style={at={(0.5,0.25)},anchor=north west}
] 

\addplot[-] file {./data/storage3mb.CF.tikzdata};
%\addlegendentry{Data} 

\end{axis} 
\end{tikzpicture} 

 }
 \vskip -0.3cm
 \caption{The implemented BDI controller and its (success) performance ($y$-axis) over the number of episodes ($x$-axis).}
 \end{center}
 \label{fig:energystorage}
 \vskip -0.3cm
 \end{figure*}

%!TEX root = ../aamas11storage.tex
% %%%%%%%%%%%%%%%%%%%%%%%%%%%%%%%%%%%%%%%%%%%%%%%%%%%
\section{Experimental Results}\label{sec:results}
% %%%%%%%%%%%%%%%%%%%%%%%%%%%%%%%%%%%%%%%%%%%%%%%%%%%

In this section, we show experiments that demostrate the suitability of the framework developed in Section~\ref{sec:framework} using the energy domain example of Section~\ref{sec:application}. In particular we report on three experiments that highlight the adaptive behaviour of the controller under different situations where the environment dynamics are changing. The first experiment shows how the agent recovers functionality against deterioration in module capacities. In the second, the agent is exposed to partial failure of the system and a variable solution space caused by some modules malfunctioning and then subsequently being restored. Finally, in the third experiment we analyse the controller response and learning behaviour in the extreme case where the system suffers complete failure for some time and is thereafter restored.

\subsubsection{Experimentation Setup}

The following experimental setup applies to all experiments. We conduct experiments for a battery system with {\em five} modules. For each module, the charge state is described by a discrete value in the range $[0:3]$ where zero indicates a fully discharged state and three indicates a fully charged state. As well as this, each module has an assigned configuration for the period from the set $\{+c, 0, -c\}$ where $c=1/n$ and $n=5$ is the number of modules in the system. The desired response is in the range $[-1.0:+1.0]$ in discrete intervals of $\pm c$. The complete state space for the problem is described by the number of modules ($5$), the possible requests ($11$), the charge state of the system ($4^5$), and the assigned configuration of the system ($3^5$), or $5 \cdot 11 \cdot 4^5 \cdot 3^5 \approx 13.7$ million. This is significant, however note that we do not have to learn over this full set because the filtering of nonsensical configurations by the plans context conditions reduces the space substantially.

At the beginning of a learning episode the configuration of each module is reset to $0$. The charge state of each module, however, is left untouched and carries over from the previous episode as would be the case in the deployed system. This has implications for learning, particularly that the state space is not sampled uniformly. Each episode corresponds to one $G$ goal-event request received from the environment. For simplicity of analysis, the environment only generates satisfiable requests given the state of the battery, such that a solution always exists for the generated request. The outcome of each episode is either no response (no configuration was executed), or a single invocation of the $Execute$ plan that operates the battery and evaluates the response. The operational model is simple, so that charging adds $+c$ while discharging adds $-c$ to a module's charge state, otherwise the state is unchanged for the period i.e there is no charge loss. The tolerance level is set to $0.0$ so that the battery response is deemed successful only when the sum of the module configurations matches the request exactly.

The applicability threshold for plan selection is set to $40\%$ so plans with a likelihood of success below this value are discarded from the applicable set. The threshold for stability calculation is set to $\epsilon=0.5$. For the stability-based metric $\C_s(\cdot,\cdot,k)$, and the world-based metric $\C_d(\cdot,k)$ the averaging window uses the last five recordings i.e. $k=5$. For the calculation of the final confidence measure $\C(\cdot,\cdot,\cdot)$ the stability-based and world-based component metrics are given an equal weighting using $\alpha=0.5$. Finally, each experiment is run five times and the reported results are averages from these runs.

\subsubsection{Experiment 1: Capacity Deterioration}

In this experiment we model the situation where module capacities deteriorate over time. In a real system this will happen gradually over a very long period, however to ease the analysis of the impact of the change we force the situation to occur instantaneously in this experiment. Figure \ref{fig:experiment1} shows the system performance about this change. In the beginning of the experiment, the system performs ideally as programmed, and goes about recording its experiences although there is no evident use of the resulting learning yet. After some time (about $5000$ episodes), a forced change in the environment causes all five modules to instantaneously drop capacity (initially in the range $[0:3]$) to $[0:2]$. This results in a rapid drop in performance corresponding to the set of programmed/learnt solutions that no longer work. The ideally programmed system would at this point converge to $\approx 76\%$ performance if this change were permanent. The learning system however, aptly rectifies the situation by learning to mostly avoid ``solutions'' that no longer work. 


%\begin{figure}[t]
%\begin{center}
%%!TEX root = ../ijcai11storage.tex
\begin{tikzpicture}

\begin{axis}[
width=0.28\columnwidth,height=3cm,scale only axis,
axis line style={-}, xtick style={-}, ytick style={-},
%xlabel=Episodes,
%ylabel=Success,
every axis y label/.style={at={(-0.12,0.5)},rotate=90,anchor=center}, 
%every axis x label/.style={at={(0.5,-0.15)},anchor=center},
grid=both,grid style={-,style=densely dotted},
legend style={at={(0.5,0.25)},anchor=north west}
] 

\addplot[-] file {./data/storage1b.CF.tikzdata};
%\addlegendentry{Data} 

\end{axis} 
\end{tikzpicture} 

%\end{center}
%\caption{Adapting to capacity deterioration.}
%\label{fig:experiment1}
%\end{figure}

\subsubsection{Experiment 2: Partial Failure with Restoration}

In this scenario, we model a series of module malfunctions and their subsequent restoration. In particular $Module1$ fails for the duration $[0:20k]$ after which it is reinstated, $Module2$ fails for the period $[20k:40k]$, and so on. Figure \ref{fig:experiment2} shows the system performance about this change.\footnote{The apparent difference in performance drops at $0k$ and $20k$ is due to the fact that the sampling of the state space is non-uniform and it just happens that more ``bad'' situations occurred about the first change than the second. The theoretical drop in performance for this change is $45\%$.} At the beginning of each change (at $0k$, and $20k$), the performance drops dramatically as the expected solutions that utilise the failed module no longer work. Following this, in each instance the system successfully learns to operate the battery without the module that is out of operation (ie. by always configuring that module to not-in-use). Note that by the time each failed module is restored, the system has already learnt to operate without it and so will not try to explore the restored possibilities until a future change causes a drop in performance and triggers new exploration.


%\begin{figure}[t]
%\begin{center}
%%!TEX root = ../ijcai11storage.tex
\begin{tikzpicture}

\begin{axis}[
width=0.5\columnwidth,height=4cm,scale only axis,
axis line style={-}, xtick style={-}, ytick style={-},
%xlabel=Episodes,
%ylabel=Success,
every axis y label/.style={-,at={(-0.12,0.5)},rotate=90,anchor=center}, 
%every axis x label/.style={at={(0.5,-0.15)},anchor=center},
grid=both,grid style={-,style=densely dotted},
legend style={at={(0.5,0.25)},anchor=north west}
] 

\addplot[-] file {./data/storage2b.CF.tikzdata};
%\addlegendentry{Data} 

\end{axis} 
\end{tikzpicture} 

%\end{center}
%\caption{Adapting to module failures/restorations over time.}
%\label{fig:experiment2}
%\end{figure}

\subsubsection{Experiment 3: Complete Failure with Restoration}

In this experiment, we model the extreme scenario of complete failure of the system for some time followed by full restoration. Here {\em all} solutions fail for the period $[0:5k]$, after which they are reinstated. Figure \ref{fig:experiment3} shows the system performance about this change. At the beginning of the experiment, performance drops to zero rapidly as all of the ideal solutions start to fail. After an initial period of failures (at around $2k$ episodes), the estimated likelihood of success of all plans drops below the applicability threshold of $40\%$. Beyond this point the battery operation comes to a complete halt since no plans are ever applicable in any situation and so the $\pExecute$ plan never gets invoked. The original behaviour is then reinstated at $5k$, however if by this stage no plans are being tried then new learning will also not occur. For this reason, the applicability threshold is implemented as a ``soft'' threshold. To be exact, the $40\%$ threshold applies $90\%$ of the time leaving some possibility for selecting plans below this threshold. This allows the battery to operate with some likelihood\footnote{Note that the battery operates after five $Set*$ plan selections. In the best case only one of these plans has failed the threshold and there is a $10\%$ chance that the battery will operate. However, if all plans fail the threshold then there is only a $0.1^5$ or $0.00001\%$ chance that the battery will operate.}, and the system is able to adapt to the new solution set beyond $6k$ episodes.

%\begin{figure}[t]
%\begin{center}
%%!TEX root = ../aamas11storage.tex
\begin{tikzpicture}

\begin{axis}[
width=0.8\columnwidth,height=4cm,scale only axis,
axis line style={-}, xtick style={-}, ytick style={-},
xlabel=Episodes,
ylabel=Success,
every axis y label/.style={-,at={(-0.12,0.5)},rotate=90,anchor=center}, 
%every axis x label/.style={at={(0.5,-0.15)},anchor=center},
grid=both,grid style={-,style=densely dotted},
legend style={at={(0.5,0.25)},anchor=north west}
] 

\addplot[-] file {./data/storage3mb.CF.tikzdata};
%\addlegendentry{Data} 

\end{axis} 
\end{tikzpicture} 

%\end{center}
%\caption{Adapting to system restoration after complete failure.}
%\label{fig:experiment3}
%\end{figure}

\Shout{Add closing para for the results section here.}
%\medskip

