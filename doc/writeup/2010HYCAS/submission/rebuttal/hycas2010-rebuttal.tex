\documentclass[preprint,12pt]{elsarticle}
\parindent 0pt
\parskip 0.33ex

\usepackage{color}
\definecolor{gray}{rgb}{0.2,0.2,0.2}
\newcounter{countshout}
\newcommand{\shout}[1]{
\vspace{0.15in}
\noindent
%\begin{center}
%\begin{tabular}{c}
		\fcolorbox{black}{white}{
			\begin{minipage}{0.96\textwidth}
			#1\addtocounter{countshout}{1}%
			\end{minipage}
			}
		\fcolorbox{gray}{gray}{\textcolor
			{white}{\textbf{Action}}
		}
%%%		{white}{\textbf{\ Note \thecountshout\ }}
%\end{tabular}
%\end{center}
\vskip 0.25in
}

\newcommand{\question}[1]{\subsection{#1}}

\begin{document}

\begin{frontmatter}

\title{Extending BDI Plan Selection to Incorporate Learning from Experience\\
(Responses to reviewers)}

\author{Dhirendra Singh \& Sebastian Sardina \& Lin Padgham}
\ead{\{dhirendra.singh,sebastian.sardina,lin.padgham\}@rmit.edu.au}
\address{RMIT University, Melbourne, Australia}

\begin{abstract}
This document details the changes based on, including rebuttal against, reviewers comments on the original manuscript submitted to the Special Issue on Hybrid Control of Autonomous Systems of the Journal of Robotics and Autonomous Systems \cite{Singh:HYCAS10}.    
\end{abstract}

\end{frontmatter}


%%%%%%%%%%%%%%%%%%%%%%%%%%%%%%%%%%%%%%%%%%%%%%%%%%%%%%%%%%%%%%%%%%%%%%%%%%%
\section*{General Comments}
%%%%%%%%%%%%%%%%%%%%%%%%%%%%%%%%%%%%%%%%%%%%%%%%%%%%%%%%%%%%%%%%%%%%%%%%%%%

We would like to point out that our previous work referred to as ``under review''
in the manuscript  is now accepted as a peer-reviewed publication
\cite{Singh:AAMAS10} and we have duly updated the citation. Some concerns raised
by the reviewers have also been addressed in \cite{Singh:AAMAS10} and we have
noted this in our responses where appropriate.

%%%%%%%%%%%%%%%%%%%%%%%%%%%%%%%%%%%%%%%%%%%%%%%%%%%%%%%%%%%%%%%%%%%%%%%%%%%
\section{Reviewer \#1 Comments and Responses}
%%%%%%%%%%%%%%%%%%%%%%%%%%%%%%%%%%%%%%%%%%%%%%%%%%%%%%%%%%%%%%%%%%%%%%%%%%%

% {\em The paper is concerned with learning plan selection for parametrized goals
% with sub-goal recursion in the context of a BDI architecture. The work is based
% on an earlier approach by the same authors applying decision-tree (DT) learning
% to this problem. Previously, goals were not parametrized and hence no recursion
% needed. Technical contributions include equations to compute a measure of
% confidence in the DT classification based on both plan and domain complexity and
% a likelihood measure for selecting a plan for execution. Experimental results
% using the Towers of Hanoi problem are presented and discussed.}

\question{ The problem is well motivated and is clearly relevant to the topic
of this special issue. The way the above two measures are computed seems plausible,
although this is only partly supported by the experiments. This is because of the
"optimizations" you mention on page 16 for the particular problem at hand. What
would happen if you did not use these optimizations? }

\label{rev1:optimisations}

We see how the details of the optimisations have added to some confusion.
% %
These optimisations were in place for \emph{all} experiments and so impacted both
approaches --- the baseline $ACL$ and the improved $ACL+\Omega$. Our main
motivation for this was to speed-up absolute convergence times for both
approaches during experimentation allowing us to significantly reduce the
run-time for our full suite of experiments.
% %
The impact of not applying the optimisations is that the graphs of Figure 5 a/b
and Figure 6 a/b take more absolute iterations to converge to the solutions. This
however does not impact any of our claims about the effectiveness of the new
$ACL+\Omega$ ``relative'' to the baseline $ACL$.



\shout{ For that reason we have reworded the paragraph to remove the details of
the optimisations per se, as they are not relevant to the discussion at hand and
have no bearing on the relative results. }


\subsection{You say on p. 12 that $\delta_{Pt}$ can be computed offline. It is
not clear to me how. In your example, you state the equation for $\delta_Pt$ but
not how you obtained it. Was it derived automatically or by hand?
}

% The idea of $\delta_Pt$ is a decay factor that matches the ``complexity'' of the
% goal-plan structure. This factor would be smaller for a less complex structure
% than for a more complex one. The goal-plan structure itself is nothing more than
% the BDI program and is given upfront in the form of a plan library that may be 
% parsed offline to compute a structure similar to that shown in Figure 2. We may
% further parse this structure to compute a plan's complexity relative to another.
% Previously \cite{Singh:AAMAS10} we have used an accurate calculation of the
% number of choices below the plan to measure this. However, the same calculation
% is not possible with parameterised goals --- we cannot compute offline the full
% set of possibilities as it depends on the runtime values of the parameters. 

For this work, we have calculated $\delta_{Pt}$ in terms of average breadth and
depth of the structure, where the depth of a structure with recursion is the
maximum level of recursion, to provide an approximation of the complexity of the
structure.
% %
Of course, there are other ways of calculating $\delta_{Pt}$ (e.g., in
\cite{Singh:AAMAS10} we have used an accurate calculation of the number of
choices below plans; however this is not feasible anymore when goals and plans
are schemas with possibly infinite instances); the whole idea is to somehow
measure the complexity of a hierarchical structure.

\shout{We have updated the related paragraph on page 12 to clarify that
$\delta_{Pt}$ is computed in terms of breadth and depth of a structure but that
that other measures may be used with analogous overall results.}




\subsection{You only compare your new approach to your previous work in the experiments. Why did you not include a comparison with the existing BDI program that comes with JACK? This is all the more surprising since you yourself ask: does our learning framework achieve the performance of the existing system. Does the existing program always do the right thing? In any case, it would be good to compare at least some of the hand-crafted context formulas with those learned by the DT approach.}

The original JACK program has perfectly crafted context conditions that work precisely so that all $Solve$ events are always successfully resolved.

The learnt context conditions are weaker than the original conditions. This is evident in Figure 6 a/b where $ACL+\Omega$ achieves about $80\%$ of the original performance. In other words, in $20\%$ of cases a plan is selected (incorrectly) to resolve a goal when it would not have been in the original case.

\shout{We have updated the discussion of results on page 20 to clarify this point.}

We would like to point out here that the learnt context conditions do not ``look'' like the original ones. When the learnt decision trees are converted to rules, they are quite complex and perhaps not intuitive as the original conditions are. In general, there is a deeper issue here to do with the fact that often in real programs control knowledge is encoded into the parameters which is not possible to learn. This discussion is not in the scope of this work. Nevertheless, we believe that the important comparison here is the performance of the original and learnt systems, not the internals.



\subsection{The paper is well written and the relevant literature is discussed/cited as far as I can tell. (I am no BDI expert.)}
\label{rev1:related_work}

\shout{Please note that we have further updated the related work account in [Section 5. Discussion and Conclusion].}

\subsection{Typo: p.20: preiously}

\shout{This has now been corrected.}




%%%%%%%%%%%%%%%%%%%%%%%%%%%%%%%%%%%%%%%%%%%%%%%%%%%%%%%%%%%%%%%%%%%%%%%%%%%
\section{Reviewer \#2 Comments and Responses}
%%%%%%%%%%%%%%%%%%%%%%%%%%%%%%%%%%%%%%%%%%%%%%%%%%%%%%%%%%%%%%%%%%%%%%%%%%%

% {\em This paper presents work that aims at adding learning capabilities to
% BDI agent systems. In particular, the authors introduce learning of
% the context conditions of BDI agent programs, that is, the state
% conditions that are required to hold in the current state of execution
% in order for an agent program to be considered for execution.
% The approach is based on replacing the context formulae of programs by
% decision trees. The work builds on previous work by the authors by
% generalizing the learning scheme to work with parametrized goal/events
% and (bounded) recursive programs. An experimental evaluation of the
% learning approach is presented showing the improvement over their
% previous work.}


\subsection{The overall work seems reasonable and appears to be effective,
although it does not seem to be particularly challenging. The learning
approach is limited by the inabilitiy to take into account goal
inter-dependencies. Nevertheless, the work represents progress towards
endowing BDI agents with learning capabilities.}

Although the problem may initially appear non-challenging, it indeed is. There
are many nuances that come up mainly namely due to the fact that \emph{(i)} the
objects to be learnt (i.e., plan's context conditions) are strongly
inter-related, what happens in one plan is affected by what happens in the plans
below; \emph{(ii)} learning and acting happens interleaved; and \emph{(iii)}
there are typically many more failures than successes.

Providing a \emph{principled} exploration mechanism of the several alternatives
an agent may have in its know-how structure turns out to be not trivial, which
explains why almost no work has been done on the topic.

With respect to the issue of inter-deendencies, we point out that this is not a
limitation of our learning approach itself, but of the overall BDI programming
paradigm. Addressing this would require a substantial modification in the BDI
programming style in terms of representation, which is out of the scope of this
paper. Here we aim to provide a learning account compatible with standard
existing BDI architectures.

 
\subsection*{There are some parts of the paper where clarity could be improved.
More detailed comments follow.}

\subsection{In sec.3, 1st para., what do you mean by "event-goal type"? just a
parametrized event-goal?}

\shout{That is correct. We have updated to manuscript to be consistent and only use one of these terms.}

\subsection{The whole section 3.1 is written in a confusing way.}

\shout{We have further reviewed and updated this section to help improve understanding.}
 
\subsection{In the 2nd para. pg 9, you talk about "the plan". One can guess that
you mean learning is applyied for each plan in the library. But it
would be better to make it explicit here, e.g., that you need a
training data set for each plan in the library.}

\shout{Yes that is the case. This has now been addressed.}

\subsection{At the end of this paragraph, it says that a classification is learnt "based purely on the subset of attributes in w that are relevant to the contex condition of the plan." This is not clear. What 'w' are you referring to here? The next para. starts again talking about "The attributes in w...", what 'w'?}

The previous paragraph explains this. Here $w$ is ``the world state in which the plan was executed'' and is comprised of ``a set of discrete attributes that together represent the state of affairs'' in the environment.


\subsection{Also in the 2nd para. it says "the decision tree will learn a
classification..." isn't the tree itself what is being learnt?}

\shout{Yes. This has now been addressed.}


\subsection{In the 3rd para. pg 11, it says that "...we may never find the bottom or leaf nodes. This has implication for any bottom-up strategies." You mean top-down strategies?}

\shout{We have now reworded this.}

\newcommand{\tuple}[1]{\langle #1 \rangle}		% tuple

\subsection{In sec. 3.2, goals G are sometimes followed by square brackets, as in
$G[\phi]$, and sometimes buy angle brackets, $G\tuple{\phi}$, which one may
guess represent modalities, but the notation has not been defined. So the reader
is left guessing what the notation means.}

We apologise for the confusing notation. These are not modalities. $G[P:w]$
states that plan $P$ is tried on world $w$ to resolve goal $G$. We have changed
$G\tuple{\phi}$ to $G(\vec{x})$ which denotes goal $G$ with parameters $\vec{x}$
(e.g., $clean(room1)$).

\shout{We have clarified and fixed the notation in Section 3.2. We now
just use $G[P:w]$ and $G(\vec{x})$.}




\subsection{Sec 3.4, 1st para., you refer to a "previous definition of the
learning task (Section 3)", which one can guess refers to the informal
definition at the beginning of Section 3, since there is no definition
anywhere. Btw, the reference itself is in section 3.}

\shout{We have now reworded this.}

\subsection{Also in sec 3.4, you define the training data set as tuples
[w U phi, o]. This is strange since I would think the world would be
represented differently from a goal's parameters, but you specify the
first element in a training data tuple as the union of a world and
the goal's parameters. Why not add a ground goal instead? Or use a
triple?}

\shout{We have now updated this to use a triple. Thank you.}

\subsection{At the end of sec. 4.1, you discuss some "optimizations" that are
possible in the case of the towers of hanoi problem. It is not really
clear why you need to make these optimizations (I'd called them
simplifying assumptions rather than optimizations). What if you don't
make them?}

We refer the reviewer to the response in \ref{rev1:optimisations} that addressed this concern.

\subsection*{Typos etc:}

\subsection{pg 8: "a plans ..." -> "a plan ..." or "a plan's ..." (occurs in many places).}

\shout{We have reviewed the manuscript and corrected these cases.}
\subsection{pg 20: "preiously learnt"}

\shout{This has now been corrected.}



%%%%%%%%%%%%%%%%%%%%%%%%%%%%%%%%%%%%%%%%%%%%%%%%%%%%%%%%%%%%%%%%%%%%%%%%%%%
\section{Reviewer \#3 Comments and Responses}
%%%%%%%%%%%%%%%%%%%%%%%%%%%%%%%%%%%%%%%%%%%%%%%%%%%%%%%%%%%%%%%%%%%%%%%%%%%

{\em The paper addresses a nice new idea of learning of conditions in the context of bdi plan hierarchy. It claims that with the approach better conditions can be learned based on the success of previous executions plans. The approach is based on previous work of the authors. The paper is well written and is easy to understand but some parts in the sections 2 and 3 would be easier to understand if a good real example would guide the line of argumentation (e.g. an example from the agents or robotics domain). The major weaknesses of the paper is the lack of emperical or formal results and the number of assumptions made which makes the general applicability questionable. detailed remarks:}


\subsection{A good example of a domain used in the explanation of the background in section 2 and in the development of the approach in section 3 would ease the understanding of the paper and would even non-bdi experts to profit from the paper.}
\label{rev3:recursion_examples}

\shout{We have now included references to several logistics domains used in the planning literature that are also relevant to our discussion. Where?...}


\subsection{In the middle part of page 7 the authors claim that event types and recursion is important to handle. But it is not very well explained why these constructs are important to handle.} 

In procedural languages that do not provide constructs for looping (for instance, $for$ and $while$ loops in programming), the basic mechanism for achieving this behaviour is through recursion. Since BDI falls in this category, then parameterised events and recursion become essential features that are heavily used in real systems.

\shout{We have added this explanation to ...}

\subsection{Also throughout the paper it is not really point out how the event types are handled in fact.}

We believe this is adequately addressed in the manuscript in [Section 3.4 Handling Event-Goal Types].

\subsection{The term "coverage" is used very often starting with page 7. It seems to be an important part in the approach but is neither formally defined nor really explained. A definition and a deeper explanation would help the reader to understand the approach.}

\shout{We have now clarified this in page 7.}

\subsection{On page 9 the authors introduce decision trees to represent the chance of a plan to be successful which is used later to define if a plan should be used in a certain situation. To me it is not clear as a decision tree covers both the negative and the positive case of the execution how the information on the negative outcome is used.}

In a given world state $w$ (i.e. a set of attributes as far as the decision tree is concerned), a plan may fail due to a poor choice of sub-plan but later pass when all correct sub-choices were made. Here, both outcomes will be recorded in the training set against world state $w$ (i.e tuples [w,F] and [w,T]). This  along with outcomes from other world states for which the plan was invoked.

The world (attributes set) and outcome (positive or negative) pairs allow the decision tree induction algorithm ($J48$ from package $weka$) to filter out attributes that carry the least information. The resulting decision tree is essentially an abstract representation to best explains the observations. 

If we were to only use the positive outcomes, we would instead get a look-up table (decision table) for success in given world states. However, using both negative and positive outcomes, we can build an abstract representation (decision tree) that may be used to predict outcomes even for ``as yet unseen'' worlds --- based on the values of only those world attributes that are deemed important by the representation.


\shout{We have updated [Section 3.1 Integrating Decision Trees into Context Conditions for Plans] to help clarify this further.}

\subsection{On the second half of page 9 the authors claim that the set of attributes used in the learning of a decision tree are selected by the programmer of the domain. This means in the first place additional work for the programmer and might in the second place prevent the system to find a good solution. Would it be feasible to learn the conditions without this artificial limitation of the attributes?}

The reviewer highlights an important issue, that of knowledge representation, that is the focus of ongoing work. Due to lack of space however, we do not delve into this in much detail apart from paragraph 2 on page 9. 

To answer your question, one way to automate the attribute set is to parse the plan bodies for this information. However, this may still miss features that are important for success but are never used in the plan body (and would normally be included in the context condition of the plan). An alternative is to include the full set of attributes, however that would entail longer learning periods for collecting sufficient data to filter out the irrelevant attributes. For this work, we suggest the domain expert make the first pass at filtering out the (certainly) irrelevant attributes.

\subsection{On page 10 the authors explain recursion and the need for. Could you give a proper example for a recursive event goal in a real world domain.}

This is addressed by the response in \ref{rev3:recursion_examples}.

\subsection{In chapter 3.3 the authors explain that the training set is build up incrementally. To my understanding it might be the case that the training set so far and the plan selection learned so far has an influence of the future training data. Therefore, is it possible that a somehow unlucky plan selection and therefore probably bad training set prevent the system to converge to a good or correct set of conditions?}

This very concern has been the subject of our previous works \cite{Airiau:IJAT09,Singh:AAMAS10} to date. In the first instance \cite{Airiau:IJAT09} we look at how the notion of informed decisions may be used to do selective recording of results in order to filter out misleading training data. In the second \cite{Singh:AAMAS10} we explore a confidence measure that may be used to  decide how much value we will put in the resulting decision tree. These are two orthogonal concerns i.e. how best to collect training data and how best to use it.

We agree this is an important question and have already devoted [Section 2.2 Learning for BDI Plan Selection] to this.

\subsection{Formula 2 on page 14. The construction of the formula looks to me a little bit ad-hoc. Is there a better justification for the formula? What are others or maybe better weight functions.}

The formula is the same as described in \cite{Singh:AAMAS10} where we also discuss an alternative weighting and how it may be used to adjust the performance of the system to suit different hierarchies. As \cite{Singh:AAMAS10} is now a peer-reviewed publication (was under review at this of our original submission) we do not believe this should be addressed again in this paper.


\subsection{In the middle part of page 16 the authors state that they do not continuous exploration once a solution is found. Why this is done or why it is sufficient? It seems to me that this assumption is against the authors arguments for the properties of their approach.}

We refer the reviewer to the response in \ref{rev1:optimisations} that addresses this exact concern.

\subsection{In 4.2. k is introduced but no real number for it in the experiments is presented nor the influence of this parameter is described. Please give more information on this.}

\shout{We have updated the related paragraph in [Section 4.2 Results].}

\subsection{At the end of page 18 the authors that the applicability threshold is set to 20\% in the second experiment. It seems to my a little bit artificial and tuned towards an ad-hoc justification for the approach. Please give further information on the performance of the compared algorithms in the context of this threshold.}

Yes, the threshold setting of is somewhat arbitrary, but consistent with that used in \cite{Singh:AAMAS10}. We do not believe it impacts the justification of the approach. The difference between the default weight ($0.5$) and the cut-off weight (threshold of $0.2$ in this case) will decide how much ``give'' we have in the exploration. If we were to set the threshold to be the same as the default weight then no exploration will result since the selection weight (from Equation 2) will always fail the threshold test. However for the range $0 \leq threshold < 0.5$, the justification holds.

\shout{We have updated the related paragraph to help clarify this further.}

\subsection{On the top of page a number of assumptions are stated. These assumption seems to limit the practical general applicability of the approach a lot. Please give a deeper justification for these assumptions and on probably solutions to them.}

We are unsure which page is being referred to as page 18 has no assumptions. Perhaps the reviewer is referring to the assumptions in [Section 5. Discussion and Conclusion] on page 20? In that instance, we believe the discussion covers the concern appropriately. 



\subsection{The coverage of related research seems to me a bit weak. Only three references on such a topic is not much. I guess there is more related work out there. Please rework these section.}

Please refer to the response in \ref{rev1:related_work} that addresses this concern.

\subsection{The result section is not very much convincing. First to me it is not clear why a simple toy domain like the towers of Hanoi justifies the usefulness of the algorithm.}

We reiterate that our end goal is to allow use of our learning framework in real applications. This toy domain captures several key issues in an idealised setting and provides a good platform for the extension of our previous work that used synthetic hierarchies \cite{Airiau:IJAT09,Singh:AAMAS10} to ones with real meaning. 

\shout{We have adjusted the final paragraph of the paper to reinforce this message.}

\subsection{Even the two experiments in this domain are made in a ad-hoc manner. For a journal this is too weak. I like to see a deeper treatment of the toy domain and to see experiments in a more advanced, more relevant or even real world domain. Otherwise I am not completely convinced about the advantages of the approach.}

Our experiments were chosen to highlight the key improvements to our approach using a similar experimental setting as before \cite{Singh:AAMAS10}. 

The first experiment (Figure 5) captures several important results that we wish to highlight. Firstly, it shows the benefits of the structured exploration of $ACL+\Omega$ over the original $ACL$ \cite{Airiau:IJAT09} that largely explores randomly when no solution has been found since all plans have an equal likelihood of success (nearly zero). The result further confirms the merits of the coverage-based exploration idea that in it's original form \cite{Singh:AAMAS10} does not scale well to recursive programs and forms motivation for this work. Secondly, the experiment reinforces the intuition that the structured $ACL+\Omega$ approach is likely to show more gains where the solution is more complex. For instance, $ACL$ only discovers $20\%$ of all solutions that require $5$ levels of recursion whereas $ACL+\Omega$ finds all such solutions. For levels $3$ and $1$, the advantages of $ACL+\Omega$ over $ACL$ are progressively less. 

The first experiment equips us to understand the performance of the two approaches in the full experiment (Figure 6a). Perhaps it is the choice of the experiment with applicability thresholds (Figure 6b) that seems ad-hoc to the reader in this context, however it is there to emphasise that the improved $ACL+\Omega$ avoids this issue in a manner  (of experiment) similar to the coverage approach \cite{Singh:AAMAS10} it lends from.

\shout{We have made changes to the second last paragraph of [Section 4.2 Results ]to help clarify this further.}
\shout{Page 19: Update paragraph about goal inter-dependence. The reason why we only achieve 80\% success is because in some cases the learnt solution is not optimal and requires more recursion than in the original program. However, since the experiment is run with bounded recursion sometimes this extra recursion results in violation of this limit leading to a forced failure.}


\begin{thebibliography}{10}

\bibitem{Airiau:IJAT09}
S.~Airiau, L.~Padgham, S.~Sardina, and S.~Sen.
\newblock Enhancing Adaptation in {BDI} Agents Using Learning Techniques.
\newblock In {\em International Journal of Agent Technologies and Systems},
2009.

\bibitem{Singh:AAMAS10}
D.~Singh, S.~Sardina, L.~Padgham, S.~Airiau.
\newblock Learning Context Conditions for {BDI} Plan Selection.
\newblock In {\em Proceedings of Autonomous Agents and Multi-Agent Systems (AAMAS)}, 2010.

\bibitem{Singh:HYCAS10}
D.~Singh, S.~Sardina, L.~Padgham.
\newblock Extending BDI Plan Selection to Incorporate Learning from Experience.
\newblock Submission under review for the Special Issue on Hybrid Control of Autonomous Systems (HYCAS), Journal of Robotics and Autonomous Systems, 2010.


\end{thebibliography}

\end{document}
