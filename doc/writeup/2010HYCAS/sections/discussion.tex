%!TEX root = ../hycas2010.tex
%%%%%%%%%%%%%%%%%%%%%%%%%%%%%%%%%%%%%%%%%%%%%%%%%%%%
\section{Discussion and Conclusion}\label{sec:discussion}
%%%%%%%%%%%%%%%%%%%%%%%%%%%%%%%%%%%%%%%%%%%%%%%%%%%%

This paper builds on earlier work that extends
the typical BDI programming framework to use \dt{}s as (part of) a
plan's context condition, with a probabilistic plan selection
mechanism that caters for both exploration and exploitation of plans.
We have shown in earlier work that due to the structure of BDI
programs, care must be taken in how learning is used, to avoid
problems in certain situations. In some cases these problems lead to
failure to learn at all, as we also show here.

In this paper we extend previous work to allow for parameterised
goals such as {\it Travel(\$from, \$to)} and also for recursion, both
of which are necessary for real applications. In doing this our
previous confidence measure which relied on a finite goal-plan tree
did not scale, so we have provided a more approximate measure, relying
on the principles that have been shown correct, but without the
limitations. This paper also takes an existing BDI program involving
parameterised goals and recursion, and evaluates our approach using
this program. By removing the existing context conditions, and then
learning the correct behaviour, we show that we are able to obtain
good (although not perfect\footnote{We would hope that when learning
is combined with programmer provided context conditions, the problems
preventing perfect learning here would be avoided.}) performance. We
also demonstrate that the naive approach to learning, that does not
account for the BDI program structure fails to learn given
some program structures and an applicability threshold.

There is still work to be done before our framework may be applied for {\it practical} on-line learning in situated agents. Firstly, the framework has not been integrated with standard BDI failure handling and recovery. Clearly
this will be needed (and is the subject of ongoing work), but we do not expect this to undermine any results described here. In fact a careful integration of
failure handling could improve the speed of learning as multiple
attempts could be made to achieve a (sub)-goal. However care needs
to be taken regarding changes to world state and possible interactions
between failed attempts and eventually successful ones.

Secondly, our use of \dt s is naive. 
For instance, currently execution data is maintained forever and 
\dt s re-built after each plan execution. Furthermore, we learn using actual 
world states, whereas an improvement would be to learn using relational world information.
While not ideal, our framework nonetheless allows us to focus on 
the nuances of learning in BDI programs first 
without worrying about the underlying techniques.

Finally, we do not detect and learn interactions between
sibling goals in the context of a particular parent; each
subgoal is treated ``locally.'' To handle such interactions, the
selection of a plan for resolving a sub-goal should also be predicated
on the goals higher than the sub-goal, that is, it should take into
account the ``reasons'' for the sub-goal. Addressing this would require substantial modification to the BDI programming style in terms of representation, which is out of the scope of this work.

The issue of combining learning and deliberative approaches for online 
decision making in BDI-like systems has not been widely addressed.
%
Hern\'andez et al. \cite{Hernandez04:Learning} give a preliminary
account of how decision trees may be induced on plan failures in order
to find alternative logical context conditions in a deterministic
paint-world example. 
%
In \cite{nguyen2006an-ad} learnt user preferences are incorporated during BDI plan selection in a dialogue manager application using a decision tree learner. %
In contrast, \cite{karim2006plans} take the approach of refining existing BDI plans or learning new plans as a sequence of recorded actions based on prescriptions provided by the domain expert.
%
In~\cite{Riedmiller01}, low level robot soccer skills are learnt offline and then used in the deliberative decision making process once deployed.
%
More recently, \cite{lokuge2007impro} give a comprehensive account of integrating learning in BDI deliberation for a real world ship berthing logistics domain. Here a neural network module is first trained offline on the available shipping port data and then used in a deployed BDI system to improve plan selection. Their results show significant improvement in berth productivity over the existing system of human operators.

A closely related area to BDI is that of hierarchical task network (HTN)
systems where task decompositions used are similar to BDI goal-plan hierarchies.
Recently, in similarly motivated work to ours, \cite{Zhuo09:Learning} 
proposed a method for learning HTN method preconditions under partial
observations. There, a set of constraints are constructed from
observed decomposition trees that are then solved \emph{offline} using
a constraint solver.
%
In contrast, in our work learning and deliberation are fully integrated in a
way that one impacts the other and the classical
exploration/exploitation dilemma applies. 

The BDI architecture has also been shown \cite{simari2006on-th} to be related to Markov Decision Processes that are heavily used for solving optimisation problems in reinforcement learning \cite{Mitchell97:ML}. A sub-area of work related to ours is hierarchical reinforcement learning \cite{barto2003recen} where task hierarchies similar to BDI are used. When the aim is to find locally optimal solutions for each sub-MDP in the hierarchy, similar issues as ours arise, such as goal inter-dependence. In general, global optimality is possible only when information is fed into the sub-task (i.e. value functions uses the entire state space), consistent with our analysis of goal inter-dependence issues. Interestingly, work by Dietterich \cite{dietterich2000hiera} also supports the use of simultaneous learning at all levels (similar to our \CL\ based approaches) instead of waiting for the children to converge (analogous to our conservative  approach \cite{Singh:AAMAS10}).

% %
Although there is still work to do before we can expect learning to be
successfully integrated into a fully autonomous BDI agent, the
work reported here is significant in that it provides a solid
foundation for adding new capabilities to BDI agents to allow them to
learn and adapt based on experience.

