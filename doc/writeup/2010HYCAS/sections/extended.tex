%%%%%%%%%%%%%%%%%%%%%%%%%%%%%%%%%%%%%%%%%%%%%%%%%%%%
\section{Extended Framework: Event Types and Recursion}\label{sec:extended}
%%%%%%%%%%%%%%%%%%%%%%%%%%%%%%%%%%%%%%%%%%%%%%%%%%%%

\notems{intro para for motivating the section in the context of sections 2 and
3} The learning framework discussed so far suffers from several serious limitations
for real world applicability.
% %
In particular, events were so far assumed propositional atoms, that is, event
types were not considered. By \emph{event type} we mean here an event that may
contain ``data'' as part of its definition. For instance, event
$\textmath{travelTo(?dest)}$ may represent the goal to travel to location
$\textmath{?dest}$ and $\textmath{moveDisk(?disk,?pin)}$ the goal to move a
certain disk to a given pin for a robot trying to solve the Towers of Hanoi game.
% %
Another limitation is the assumption that the agent's plan library is
non-recursive, so that the goal-plan tree structure induced is always
\emph{finite}.
% %
Clearly both limitations would preclude the applicability of the approach in
most real domains. Real world hierarchical structures encoding know-how information
are generally expressed using parametrized goal events and plans, and often make
use of (direct or indirect) recursive procedures.



\subsection{Event Types}



\subsection{Recursion}



