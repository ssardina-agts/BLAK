%%%%%%%%%%%%%%%%%%%%%%%%%%%%%%%%%%%%%%%%%%%%%%%%%%%%%%%%%%%%%%%%%%
%%%%%%%% ICML 2010 EXAMPLE LATEX SUBMISSION FILE %%%%%%%%%%%%%%%%%
%%%%%%%%%%%%%%%%%%%%%%%%%%%%%%%%%%%%%%%%%%%%%%%%%%%%%%%%%%%%%%%%%%

% Use the following line _only_ if you're still using LaTeX 2.09.
%\documentstyle[icml2010,epsf,natbib]{article}
% If you rely on Latex2e packages, like most moden people use this:
\documentclass{article}

% For figures
\usepackage{graphicx} % more modern
%\usepackage{epsfig} % less modern
\usepackage{subfigure} 

% For citations
\usepackage{natbib}

% For algorithms
\usepackage{algorithm}
\usepackage{algorithmic}

\usepackage{pgf,pgfpages,tikz}
\usetikzlibrary{calc,arrows,shadows,fit,shapes,matrix,automata,positioning,backgrounds,trees,fit,plotmarks}
\pgfdeclarelayer{background} \pgfdeclarelayer{foreground}
\pgfsetlayers{background,main,foreground}

\newcommand{\BUL}{\textsf{\small BUL}}
\newcommand{\CL}{\textsf{\small ACL}}
\newcommand{\T}{$\cal T$}

% As of 2010, we use the hyperref package to produce hyperlinks in the
% resulting PDF.  If this breaks your system, please commend out the
% following usepackage line and replace \usepackage{icml2010} with
% \usepackage[nohyperref]{icml2010} above.
\usepackage{hyperref}

% Packages hyperref and algorithmic misbehave sometimes.  We can fix
% this with the following command.
\newcommand{\theHalgorithm}{\arabic{algorithm}}

% Employ the following version of the ``usepackage'' statement for
% submitting the draft version of the paper for review.  This will set
% the note in the first column to ``Under review.  Do not distribute.''
\usepackage{icml2010} 
% Employ this version of the ``usepackage'' statement after the paper has
% been accepted, when creating the final version.  This will set the
% note in the first column to ``Appearing in''
% \usepackage[accepted]{icml2010}


% The \icmltitle you define below is probably too long as a header.
% Therefore, a short form for the running title is supplied here:
%\icmltitlerunning{Learning to refine the preconditions of plans in a plan hierarchy.}
\icmltitlerunning{Learning Dynamic Plan Selection for BDI Hierarchy} 

\begin{document}
\twocolumn[
%\icmltitle{Learning to refine the preconditions of plans in a plan hierarchy.}
\icmltitle{Learning Dynamic Plan Selection for BDI Hierarchy} 

% It is OKAY to include author information, even for blind
% submissions: the style file will automatically remove it for you
% unless you've provided the [accepted] option to the icml2010
% package.
\icmlauthor{}{}
\icmladdress{}
\icmlauthor{}{}
\icmladdress{}

% You may provide any keywords that you 
% find helpful for describing your paper; these are used to populate 
% the "keywords" metadata in the PDF but will not be shown in the document
\icmlkeywords{boring formatting information, machine learning, ICML}

\vskip 0.3in
]

\begin{abstract} 

Belief-Desire-Intention (BDI) paradigm provides a powerful/popular framework for building complex agent based systems.  The actions of these BDI agents are guided by a hierarchical plan tree on which every goal of actions associate with a set of plans.  Selecting a suitable plan for achieving a particular goal can be challenging especially if the environment is dynamic.  In the paper we present a hierarchical structure to learn BDI plan selection during plan execution.  Each selection point contains a separate decision tree which is built based on current experience and will be constantly updated when more experience is available.  These decision points are not independent as the decision made at one point may affect the learnable experience of subsequent points on the hierarchy(??).  Two variations of the learning method are investigated.  The results show that they not only can work coherently with BDI framework, but also can take advantage of the failure recovery mechanism in BDI to reduce the cost of learning.  Furthermore the success of goal/actions can be maintained even when the environment contains certain degrees of nondeterministic elements.

\end{abstract} 

\section{Introduction}

a symbolic reasoning perspective

One difficulty for practical use of software agents is to 
precise the
conditions under which a specific plan succeeds. A programmer should
be able to encode some basic preconditions of an action. Considering
the other actions needed for achieving a plan, a programmer may be
able to propagate some of these conditions for the plan preconditions.
This task, especially in a dynamic environment, may not be
trivial. Moreover, some conditions may be specific to certain
environments, which makes it very difficult for a programmer to come
up with exact and accurate preconditions for each plan.  One desirable
solution is 1) for a programmer to encode the basic preconditions
ensuring that the agent functions under a generic environment and 2)
to use a learning algorithm that automatically refines these
preconditions using the experience of the agent with its
environment. In this paper, we are interested in the online and
incremental learning of precondition of plans.

In order for the agent to be reactive, it may be useful to decompose the planning problem into sub-plans. This is for an example an idea used in Hierarchical Task Network (HTN) planning or for Belief Desire Intentions (BDI) agents.  For example, in the case of BDI agents, a plan can be decomposed into a set of sub-goals that have to be achieved in parallel or in sequence. Each of these sub-goals can be achieved by a set of sub-plans. The decomposition continues until a sub-plan coincides with an action that can be performed by an agent. One solution for learning the context condition is to use execution traces and learn the context conditions of top level plans. This solution, however, does not use the decomposition, i.e., some knowledge about the environment and the tasks. Our challenge is to learn a precondition for each plan in the hierarchy.  In other words, each plan in the hierarchy is associated with a learning algorithm that will learn to refine the context condition of the corresponding plan. This choice brings an interesting challenge. Consider for example that a plan failed because of the failure of one of its sub-plan. What went wrong: the choice of the top-level plan or the choice of its sub-plan?  It is possible that the use of a different sub-plan would have achieved the sub-goal, in which case it was correct to use the top-level plan. But it was equally possible that all sub-plans were doomed to fail and that the top level plan should not have been chosen at the first place.  For the learning algorithm, this means that the input of algorithm of the top level plan depends on the output of the lower level plan. 

Through its experience with its environment, the agent collects additional instances and may learn the right choices of plans. We consider two solutions to this learning problem and provide empirical results using induction of decision tree to learn the preconditions of the plans of a BDI agent.


\section{Related work}

One related domain is the Hierarchical Task Network (HTN) planning. Here a task can be decomposed into an ordered sequence of sub-tasks.  The decomposition of the tasks, as decomposition of plans, contains much expert knowledge about the environment and the ability of solving problems in this environment. Instead of using expert to design the HTN structure, it is interesting to learn the relationship between tasks automatically from execution traces~\cite{Zhuo09:Learning,Ilghami05:Learning,Hogg08:htnmaker}. For example, one can monitor human activities to build a database of traces, and use HTN learning algorithm to learn the task decomposition. In our work, we assume that the decomposition of plans is already been programmed using expert knowledge. However, some additional plan preconditions that depends on the particular environment may still need to be learnt. In addition, these HTN learners are to be used offline when our work deals with learning as the agents operates, i.e., from the interaction with its environment, the agent learns how to improve its performance.

Our work builds on the work in~\cite{Singh10:Learning} where two learning approaches to learn the preconditions of plans for BDI agents are presented.  One assumption from that work is that the failure recovery of the BDI agent is deactivated, i.e., when an action fails, the top-level plan fails and the agent restart to achieve the top-level goal from scratch. Failure recovery would enable the agent to try a different plan when a sub-plan has failed, i.e., the agent tries to fix the error on the go, which may be a much more effective behaviour in many environments. Using failure recovery adds additional issues to learning the preconditions of the plans, which will be described in the next section. Finally, the experiments of ~\cite{Singh10:Learning} explored different structure of the plan hierarchy, but little was reported about the plans that were learnt. In this paper, we use some classification problems from the machine learning repository to provide sounder results.


\section{BDI agents}

BDI agent-oriented programming is a popular, well-studied, and practical paradigm
for building intelligent agents situated in complex and dynamic environments with
(soft) real-time reasoning and control requirements
\cite{Georgeff89-PRS,Benfield:AAMAS06}.

In a BDI-style system, an agent consists, basically, of a belief base (akin to a
database), a set of recorded pending goal events, a plan library, and an
intention base. While the belief base encodes the agent's knowledge about the
world, the pending events stand for the \emph{goals} the agent wants to
achieve/resolve.
% %
The \textit{plan library}, in turn, contains \emph{plan rules}, or simply
\emph{plans}, of the general form $e: \psi \leftarrow \delta$ encoding the
standard domain operational procedure $\delta$ (that is, a program) for achieving
the event-goal $e$ when the so-called \textit{context condition} $\psi$ is
believed true---program $\delta$ is a reasonable strategy to resolve event $e$
whenever $\psi$ holds. Among other operations, the plan-body program $\delta$
will typically include the execution of actions ($act$) in the environment and
subgoal events ($!e$) that ought to be in turn resolved by selecting suitable
plans for that subgoal event. Lastly, the intention base accounts for the
current, partially instantiated, plans that the agent has already committed to in
order to handle or achieve some event-goal.


The basic reactive goal-oriented behavior of BDI systems involves the system
responding to events, the inputs to the system, by committing to handle one
pending event-goal, \textit{selecting a plan} from the library, and placing its
program body  into the intention base (see Figure~\ref{fig:BDI_description}).
% %%
A plan may be selected if it is \textit{relevant} and \textit{applicable}, that is, if it
is a plan designed for the event in question and its context condition is
believed true, respectively.
% %
In contrast with traditional planning, execution happens at each step. The
assumption is that the use of plans' context-preconditions to make choices as
late as possible, together with the built-in goal-failure mechanisms, ensures
that a successful execution will eventually be obtained while the system is
sufficiently responsive to changes in the environment.

\begin{figure}
\input{Figures/bdi_description}
\caption{BDI architecture}
\label{fig:BDI_description}
\end{figure}


For the purposes of this paper, we shall mostly focus on the plan library, as we
investigate ways of learning how agents can make a better use of it over time.
% %
It is not hard to see that, by grouping together plans responding to the same
event type, the plan library can be seen as a set of \emph{goal-plan tree}
templates: a goal (or event) node has children representing the alternative
relevant plans for achieving it; and a plan node, in turn, has children nodes
representing the subgoals (including primitive actions) of the plan.
% %%
These structures, can be seen as AND/OR trees: for a plan to succeed all the
subgoals and actions of the plan must be successful (AND); for a subgoal to
succeed one of the plans to achieve it must succeed (OR).

Consider, for instance, the structure depicted in
Figure~\ref{fig:T3}.
% %%
A link from a goal to a plan means that this plan is relevant (i.e., potentially
suitable) for achieving the goal (e.g., $P_1 \ldots P4$ are the relevant
plans for event goal $G$); whereas a link from a plan to a goal means that the
plan needs to achieve that goal as part of its (sequential) execution (e.g., plan
$P_A$ needs to achieve goal $G_{A1}$ first and then $G_{A2}$).
% %
For compactness, an edge with a label $\times n$ states that there are $n$ edges
of such type.
% %
Leaf plans directly interact with the environment and so, in a given world state,
they can either succeed or fail when executed; this is marked accordingly in the
figure \emph{for some particular world} (of course, in other states, such plans
may behave differently).
% %
In some world, given successful completion of $G_A$ first, the agent may achieve goal $G_B$ by selecting and
executing $P_B$, followed by selecting and executing $2$ leaf working plans to
resolve goals $G_{B1}$ and $G_{B2}$. If the agent succeeds with goals $G_{B1}$
and $G_{B2}$, then it succeeds for plan $P_B$, achieving thus goal $G_B$ and the
top-level goal $G$ itself. There is no possible successful execution, though, if
the agent decides to carry on any of the three plans labelled $P_{B2}'$ for
achieving the low-level goal $G_{B2}$.

\begin{figure}[t]
\begin{center}
\begin{tikzpicture}[scale=0.8,level distance=1.1cm]
\tikzstyle{txt}=[scale=.9]
\tikzstyle{succ}=[label=below:$\surd$]
\tikzstyle{fail}=[label=below:$\times$]

\tikzstyle{planbox}=[draw,minimum height=0.55cm,minimum width=0.55cm]
\tikzstyle{goalbox}=[draw,rounded corners,minimum height=0.55cm,minimum width=0.55cm]

	
\tikzstyle{level 1}=[sibling distance=4.0cm] 
\tikzstyle{level 2}=[sibling distance=4cm] 
\tikzstyle{level 3}=[sibling distance=2cm]
\tikzstyle{level 4}=[sibling distance=2.5cm]
\tikzstyle{level 5}=[sibling distance=1.3cm]

\node[goalbox,yshift=1cm,solid] (T) {$G$}
	child[dashed] {node[planbox] (P1) {$P_1$}}
	child[solid] {node[planbox] (Pi) {$P_i$}
		child {node[goalbox] {$G_A$}
			child {node[planbox] {$P_{A}$}
				child {node[goalbox] {$G_{A1}$}
					child {node[planbox,succ] {$\phantom{P}$}}
					child {node[planbox,fail] {$\phantom{P}$} 
					edge from parent node[txt,right,near start] {$\times 3$}
					}
				}
				child {node[goalbox] {$G_{A2}$}
					child {node[planbox,succ] {$\phantom{P}$}}
					child {node[planbox,fail] {$\phantom{P}$} 
						edge from parent node[txt,right,near start] {$\times 3$}
					}
				}
			}
			child {node[planbox,fail] {$\phantom{P}$} 
				edge from parent node[txt,right,near start] {$\times 3$}
			}
		}
		child {node[goalbox] {$G_{B}$}
			child {node[planbox,fail] {$\phantom{P}$}
				edge from parent node[txt,left,near start] {$\times 3$}
			}
			child {node[planbox] {$P_B$}
				child {node[goalbox] {$G_{B1}$}
					child {node[planbox,succ] {$\phantom{P}$}}
					child {node[planbox,fail] {$\phantom{P}$}
						edge from parent node[txt,right,near start] {$\times 3$}
					}
				}
				child {node[goalbox] {$G_{B2}$}
					child {node[planbox,succ] {$P_{B2}$}}
					child {node[planbox,fail] {$P_{B2}'$}
						edge from parent node[txt,right,near start] {$\times 3$}
					}
				}
			}
		}
	}
	child[dashed] {node[planbox] (P4) {$P_4$}}
;
\node[txt] at ($ (P1)!.5!(Pi) $) {$\ldots$};
\node[txt] at ($ (Pi)!.5!(P4) $) {$\ldots$};

\end{tikzpicture}

\end{center}
\caption{Goal-plan hierarchy $\T_3$. There are $2^4$ worlds whose solutions are
distributed evenly in each of the $4$ top level plans. Successful execution
traces are of length $4$. Within each sub-tree $P_i$, \BUL\ is expected to
perform better for a given world, but it suffers in the number of worlds. Overall, \CL\ and \BUL\
perform equally well in this structure.}
\label{fig:T3}
\end{figure}


As one can easily observe, the problem of \textit{plan-selection} is at the core
of the whole BDI approach:
\emph{which plan should the agent commit to in order to achieve a certain goal?}
% %
This problem amounts, at least partly, to what has been referred to as
\emph{means-end analysis} in the agent foundational literature
\cite{Pollack92-IRMA,Bratman88}, that is, the decision of \textit{how} goals are
achieved.
% %%
To tackle the plan-selection task, state-of-the-art BDI systems leverage domain
expertise by means of the context conditions of plans. However, crafting fully
correct context conditions at design-time can, in some applications, be a
demanding and error-prone task. In addition, fixed context conditions do not
allow agents to adapt to changing environments.
% %%
In the rest of the paper, we shall provide an extended framework for BDI agent
systems that allows agents to learn/adapt plans' context conditions, and discuss
and empirically evaluate different approaches for such learning task.


\section{Learning problem and two learning approaches}

Our aim is to add learning capabilities to adapt and to refine context
conditions of plans to the particular environment where the agent
acts.  A first step in this direction is to use a decision tree (DT)
in addition to the context condition. A decision tree is suitable for
this task.  First, the state of the world can be represented using
attribute-value pairs, since it is an implicit assumption of the
implementation of BDI agents. Next, the issue at hand is a
classification problem: is the plan applicable in the current state of
the world?  Then, the decision tree can represent any context
condition of standard BDI agent, since such a context condition is a
formular of propositional logic. Finally, an agent may act in a non
deterministic environment, and induction of decision trees is able to
handle some noisy data.  Note, however, that we intend to use
induction of decision trees in an incremental and online setting,
which is not a standard use.  As the agent interacts with its
environment, new instances are added to the data set of each decision
tree. In the experimental section, we will induce a new decision tree
every time an instance is added to the data set. For practical use of
our agents, we intend to use algorithm that induce a decision tree in
an incremental way~\cite{Swere06:Fast,Utgoff97Decision}.

\subsection{Learning space}

For each plan in the goal-plan tree of the BDI agent, we associate a
decision tree. We assume that a programmer provides, for each plan, a
set of attributes that may play a role in the success of a plan. Let
us assume that the success of the plan $P$ is function of the values
of the attributes in the set $D^{\top}$. We make the assumption that,
the set of attributes $D$ provided by the programmer is such that
$D^{\top} \subseteq D$. If the programmer may not be able to formalize
precisely the precondition of $P$, he should at least be able to tell
a superset of the set $D^{\top}$.

Let us consider a plan $P$ with subplans $p_1$, $\dots$, $p_n$, and
let further assume that $d_i$ is the set of attributes provided by the
programmer for sub-plan $p_i$. The set of attributes $D$ to be used
for $P$ is such that $D \subseteq \cup_{i=1}^n d_i$. The success of
$P$ depends on the values of each $d_i$, but not necessarily all. Let
$a \in d_i$ for a given $i$. It is possible to select a plan that will
succeed for all possible values of $a$, in which case the success of
plan $P$ may not depend on $a$. Moreover, an attribute $b \notin
\cup_{i=1}^n d_i$ should not have any impact on the success of $P$,
since it has no impact on any of its sub-plans. In theory, it is then
possible that the higher a plan in the goal-plan tree is, the larger
the learning space. We believe that the expert knowledge of the
programmer is able to keep the learning space for top level plan to a
reasonable size.

\subsection{Probabilistic Plan Selection}

The work of~\cite{Singh10:Learning} proposes to use a probabilistic
plan selection. Let us assume that a given plan $P$ has $n$ sub-plans
$p_1$, $\dots$, $p_n$ with their corresponding decision trees $d_1$,
$\dots$, $d_n$, and let us assume that the state of the world,
represented by a vector of attribute-value pairs, is $w$. For each
decision tree $d_i$, we retrieve the number of positive and negative
instances (respectively $o^+_i$ and $o^-_i$ that are contained in the
leaf node of $d_i$ corresponding to $w$. For decision tree $d_i$, we
compute the frequency of success $f_i$ of the node that contains $w$:
$f_i=\frac{o^+_i}{o^+_i + o^-_i}$.  We then select probabilistically
plan $p_i$ with a probability proportional to $f_i$. This plan
selection will ensure some level of exploration and will favour plans
that have a higher frequency of success\footnote{Note for us: Maybe
  here, we could think for the future to improve that. We could take
  into account the size of the space represented by the leaf node. Say
  we have 50\% of success for that node. If it represents a huge space
  say 128 worlds or only 4 worlds, this 50\% may have different
  strength. We could have a bias towards exploring larger spaces}.

\subsection{Two learning approaches}

Let us first consider a plan which is a leaf node of a goal-plan tree,
i.e., a plan that corresponds to an action of the agent. For this
case, the learning problem is a standard classification problem: the
outcome of the action depends only on the environment.  The agent
records the state of the world, acts, records the outcome of the
action, and updates the decision tree. When this plan is used a
sufficiently large number of times, the induction algorithm will learn
a correct decision tree.

Now, let us consider a plan $P$ that is not a leaf node of a goal-plan
tree. This plan has at least a sub-plan. Unlike in the previous case,
the input of the decision tree of $P$ will not only depend on the
environment: it also depends on the choice made for the
sub-plan(s). If $P$ succeeded, it certainly means that the correct
choices were made for the sub-plans, hence we should add the
corresponding instance in the database. In case of failure, it is not
so clear. If we knew that the choices made below in the goal-plan tree
were the \emph{correct} choices, then a failure of $P$ should be
recorded. However, if we are not that the right choices were made, it
is not clear whether the instance is a false negative or not.

First, let us consider the case of failure recovery. Say one of the
$p_i$, trying to achieve the subgoal $g$, failed.  In that case, we
claim that we should not add any instance in the data set of $P$, and
that, even if the failure is recovered.  We assume that the goal-plan
tree is correct. Consequently, if the execution needed to backtrack in
the goal-plan tree, it was not the correct, expected behavior of the
agent, hence, we should not learn it\footnote{Note that if we do not
  trust the goal-plan tree, it could be interesting to use these
  backtracks in the goal-plan tree to add new sub-trees, but we keep
  this issue for future research}.


In order to discuss further \emph{which} data should be recorded
\emph{where}, we define the notion of an \textit{active execution
  trace}, as a sequence of the form $G_0[P_0:w_0] \cdot G_1[P_1:w_1]
\cdot \ldots \cdot G_n[P_n:w_n]$, which represents the sequence of
currently active goals, along with the plans which have been selected
to achieve each of them, and the world state in which the selection
was made---plan $P_i$ has been selected in world state $w_i$ in order
to achieve the $i$-th active subgoal $G_i$.

Two mechanisms for learning were introduced
in~\cite{Singh10:Learning}. The idea of the first approach is to
filter the instances that are included in a data set by estimating the
confidence of the decision trees of sub-plans. With this approach,
decision trees of leaf plans will be learnt first, and when they
appear accurate, decision trees of the level above will start to
learn. Consequently, this method has been called Bottom Up learning
(BUL). This method requires that all the decision trees that are in
any sub-plans of the current plan to appear accurate, which is
demanding.

The idea of the second mechanism is to be less careful about which
instances are recorded in the data set of a decision tree. Instead,
Singh et al. prescribe the use of a more intelligent selection
function. This function will select probabilistically a plan with a
probability proportional to the fraction of paths that have been
tried.

\section{Results}

\begin{figure*}[t]
\begin{center}
\resizebox{0.8\textwidth}{!}{
%%%% Document setup
\documentclass[a4paper,landscape,title page]{article}
\setlength{\oddsidemargin}{-0.65in}	% default=0in
\setlength{\textwidth}{11in}		% default=9in
\setlength{\textheight}{6.85in}		% default=5.15in
\setlength{\topmargin}{-1.0in}		% default=0.20in
\setlength{\headsep}{0.35in}		% default=0.35in
\setlength{\parskip}{1.2ex}
\setlength{\parindent}{0mm}

%%%% Use packages
\usepackage{times}
\usepackage{tikz,pgf}
\usetikzlibrary{arrows,automata,trees,plotmarks,calc}


%%%%
\title{Goal-Plan hierarchy for test \textit{testfr02}}
\author{
Dhirendra Singh\\
dhirendra.singh@rmit.edu.au}
\begin{document}
\maketitle

\begin{figure*}[t]
\begin{center}

\begin{tikzpicture}[scale=1.1,level distance=1.4cm]
\tikzstyle{txt}=[scale=1.1]
\tikzstyle{planbox}=[scale=1.1,draw,minimum height=0.7cm,minimum width=1.1cm]
\tikzstyle{goalbox}=[scale=1.1,draw,rounded corners,minimum height=0.7cm,minimum width=1.1cm]
\tikzstyle{level 1}=[sibling distance=4.0cm] 
\tikzstyle{level 2}=[sibling distance=6.0cm] 
\tikzstyle{level 3}=[sibling distance=3.0cm]
\tikzstyle{level 4}=[sibling distance=3.0cm]
\tikzstyle{level 5}=[sibling distance=1.5cm]
\tikzstyle{level 6}=[sibling distance=3.0cm]
\tikzstyle{level 7}=[sibling distance=1.5cm]

\node[goalbox,yshift=1cm,solid] (T) {$G$}
	child[solid] {node[planbox] (P1) {$P_1$}
		child {node[goalbox] {$G_{a}$}
			child {node[planbox] {$P_{a1}$}
				child {node[goalbox] {$G_{aa}$}
					child {node[planbox] {$P_{aa1}$}
						child {node[goalbox] {$G_{aaa}$}
							child {node[planbox] {$P_{aaa1}$} node[txt,below=0.3cm] {$\surd$}}
							child {node[planbox] {$P_{aaa*}$} node[txt,below=0.3cm] {$\times$}
								edge from parent node[txt,right,near start] {$\times 2$}
							}
							edge from parent node[txt,right] {$\times 2$}
						}
					}
					child {node[planbox] {$P_{aa*}$} node[txt,below=0.3cm] {$\times$}
						edge from parent node[txt,right,near start] {$\times 3$}
					}
				}
				child {node[goalbox] {$G_{ab}$}
					child {node[planbox] {$P_{ab1}$} node[txt,below=0.3cm] {$\surd$}}
					child {node[planbox] {$P_{ab*}$} node[txt,below=0.3cm] {$\times$}
						edge from parent node[txt,right,near start] {$\times 2$}
					}
				}
			}
			child {node[planbox] {$P_{a*}$} node[txt,below=0.3cm] {$\times$}
				edge from parent node[txt,right,near start] {$\times 3$}
			}
		}
		child {node[goalbox] {$G_{b}$}
			child {node[planbox] {$P_{b1}$}
				child {node[goalbox] {$G_{ba}$}
					child {node[planbox] {$P_{ba1}$}
						child {node[goalbox] {$G_{baa}$}
							child {node[planbox] {$P_{baa1}$} node[txt,below=0.3cm] {$\surd$}}
							child {node[planbox] {$P_{baa*}$} node[txt,below=0.3cm] {$\times$}
								edge from parent node[txt,right,near start] {$\times 2$}
							}
							edge from parent node[txt,right] {$\times 2$}
						}
					}
					child {node[planbox] {$P_{ba*}$} node[txt,below=0.3cm] {$\times$}
						edge from parent node[txt,right,near start] {$\times 3$}
					}
				}
				child {node[goalbox] {$G_{bb}$}
					child {node[planbox] {$P_{bb1}$} node[txt,below=0.3cm] {$\surd$}}
					child {node[planbox] {$P_{bb*}$} node[txt,below=0.3cm] {$\times$}
						edge from parent node[txt,right,near start] {$\times 2$}
					}
				}
			}
			child {node[planbox] {$P_{b*}$} node[txt,below=0.3cm] {$\times$}
				edge from parent node[txt,right,near start] {$\times 3$}
			}
		}
		child {node[goalbox] {$G_{c}$}
			child {node[planbox] {$P_{c1}$}
				child {node[goalbox] {$G_{ca}$}
					child {node[planbox] {$P_{ca1}$}
						child {node[goalbox] {$G_{caa}$}
							child {node[planbox] {$P_{caa1}$} node[txt,below=0.3cm] {$\surd$}}
							child {node[planbox] {$P_{caa*}$} node[txt,below=0.3cm] {$\times$}
								edge from parent node[txt,right,near start] {$\times 2$}
							}
							edge from parent node[txt,right] {$\times 2$}
						}
					}
					child {node[planbox] {$P_{ca*}$} node[txt,below=0.3cm] {$\times$}
						edge from parent node[txt,right,near start] {$\times 3$}
					}
				}
				child {node[goalbox] {$G_{cb}$}
					child {node[planbox] {$P_{cb1}$} node[txt,below=0.3cm] {$\surd$}}
					child {node[planbox] {$P_{cb*}$} node[txt,below=0.3cm] {$\times$}
						edge from parent node[txt,right,near start] {$\times 2$}
					}
				}
			}
			child {node[planbox] {$P_{c*}$} node[txt,below=0.3cm] {$\times$}
				edge from parent node[txt,right,near start] {$\times 3$}
			}
		}
	}
;

\end{tikzpicture}

\vskip 1.5cm
\caption{The hierarchy has 4 top level plans $P_1 \ldots P_4$ and the total number of worlds is $2^5$. Only $P_1$ has the solutions, $P_2 \ldots P_4$ are all single action plans that always fail. Successful execution trace is of length $9$ distributed between goals $G_a \ldots G_c$. The aim is to compare how many actions it takes on average for the top level goal $G$ to succeed --- with and without failure recovery. The intuition is that failure recovery should help learn $G$ faster. For instance, say we have learnt to achieve $G_a$ and $G_b$. Then to learn $G_c$ we must first perform $G_a$, then $G_b$ and then try different options as we acquire experience in $G_c$. If we were to re-post $G$ after each failure then for each unsuccessful choice in $G_c$ we have to do a lot of re-work (i.e. achieve $G_a$ and $G_b$ again) before we can try something else. With failure recovery enabled however, when learning $G_c$ we only perform $G_a$ and $G_b$ once and then exhaust all options for $G_c$ in a single try. However, this assumes that failures do not (generally) change the world in such a way that alternatives tried with failure recovery will never succeed. If that were the case then failure recovery would perform worse than reposting.}
\end{center}
\end{figure*}

\end{document}
%%%%
}
\end{center}
\caption{The hierarchy has 4 top level plans $P_1 \ldots P_4$ and the total number of worlds is $2^5$. Only $P_1$ has the solutions, $P_2 \ldots P_4$ are all single action plans that always fail. Successful execution trace is of length $9$ distributed between goals $G_a \ldots G_c$. The aim is to compare how many actions it takes on average for the top level goal $G$ to succeed --- with and without failure recovery. The intuition is that failure recovery should help learn $G$ faster. For instance, say we have learnt to achieve $G_a$ and $G_b$. Then to learn $G_c$ we must first perform $G_a$, then $G_b$ and then try different options as we acquire experience in $G_c$. If we were to re-post $G$ after each failure then for each unsuccessful choice in $G_c$ we have to do a lot of re-work (i.e. achieve $G_a$ and $G_b$ again) before we can try something else. With failure recovery enabled however, when learning $G_c$ we only perform $G_a$ and $G_b$ once and then exhaust all options for $G_c$ in a single try. However, this assumes that failures do not (generally) change the world in such a way that alternatives tried with failure recovery will never succeed. If that were the case then failure recovery would perform worse than reposting.}
\label{fig:test01fr}
\end{figure*}

\begin{figure*}[t]
\begin{center}
\subfigure[ACL]{\label{fig:result_1a}
\begin{tikzpicture}[x=0.0068cm,y=0.05cm]
    % Draw the axes and grid lines
    \draw[-] (0,0) -- (0,100) -- (1000,100) -- (1000,0) -- cycle; 
    \draw[-,thin, dotted, ystep=25, xstep=1000] (0,0) grid (1000,100);
    \foreach \x in {250, 500, 750}  \draw [-,xshift=0](\x,4pt) -- (\x,-1pt);
    \foreach \y in {0,25,50,75,100}  \draw [-,yshift=0](4pt,\y) -- (1pt,\y);
    \foreach \x/\xtext in {250, 500, 750} \node at (\x,0) [below] {$\x$};
    \foreach \y/\ytext in {0,25,50,75,100}  \node at (0,\y) [left] {$\y$};
    \node at (0,115) {Actions};
    \node at (900,20) {Episodes};
    \draw[-] plot[mark=x,mark size=4,mark options={color=black}] 
			file {Data/testfr01a.CP.tikzdata};
    \draw[-] plot[mark=o,mark size=4,mark options={color=black}] 
			file {Data/testfr01b.CP.tikzdata};
    % Also draw the expected convergence: 0.9 actions
    \draw[dashed,-,yshift=0](0,9) -- (1000,9);

\end{tikzpicture}

}
\qquad
\subfigure[BUL]{\label{fig:result_1b}
\begin{tikzpicture}[x=0.0136cm,y=0.0833cm]
    % Draw the axes and grid lines
    \draw[-] (0,0) -- (0,60) -- (500,60) -- (500,0) -- cycle; 
    \draw[-,thin, dotted, ystep=15, xstep=500] (0,0) grid (500,60);
    \foreach \x in {125, 250,375}  \draw [-,xshift=0](\x,4pt) -- (\x,-1pt);
    \foreach \y in {0,15,30,45,60}  \draw [-,yshift=0](4pt,\y) -- (1pt,\y);
    \foreach \x/\xtext in {125, 250,375} \node at (\x,0) [below] {$\x$};
    \foreach \y/\ytext in {0,15,30,45,60}  \node at (0,\y) [left] {$\y$};
    \node at (0,65) {Actions};
    \node at (440,4) {Episodes};
    \draw[-] plot[mark=x,mark size=4,mark options={color=black}] 
			file {Data/testfr01a.SP.tikzdata};
    \draw[-] plot[mark=o,mark size=4,mark options={color=black}] 
			file {Data/testfr01b.SP.tikzdata};
    % Also draw the expected convergence: 0.9 actions
    \draw[dashed,-,yshift=0](0,9) -- (500,9);

\end{tikzpicture}

}
\caption{Agent performance under \BUL\ and \CL\ schemes. Circles represent performance without failure recovery while crosses show the performance with failure recovery enabled.
Each point represents results from $5$ experiment runs using an averaging window
of $5$ samples. The dashed line represents optimal performance ($9$ actions). Actions are deterministic (no noise) and failed actions do not change the world in a negative way i.e. success is always possible with recovery.}
\end{center}
\end{figure*}

\begin{figure*}[t]
\begin{center}
\subfigure[ACL]{\label{fig:result_1a}
\begin{tikzpicture}[x=0.0068cm,y=0.025cm]
    % Draw the axes and grid lines
    \draw[-] (0,0) -- (0,200) -- (1000,200) -- (1000,0) -- cycle; 
    \draw[-,thin, dotted, ystep=50, xstep=1000] (0,0) grid (1000,200);
    \foreach \x in {250, 500, 750}  \draw [-,xshift=0](\x,4pt) -- (\x,-1pt);
    \foreach \y in {0,50,100,150,200}  \draw [-,yshift=0](4pt,\y) -- (1pt,\y);
    \foreach \x/\xtext in {250, 500, 750} \node at (\x,0) [below] {$\x$};
    \foreach \y/\ytext in {0,50,100,150,200}  \node at (0,\y) [left] {$\y$};
    \node at (0,225) {Actions};
    \node at (900,20) {Episodes};
    \draw[-] plot[mark=x,mark size=4,mark options={color=black}] 
			file {Data/testfr01na.CP.tikzdata};
    \draw[-] plot[mark=o,mark size=4,mark options={color=black}] 
			file {Data/testfr01nb.CP.tikzdata};
    % Also draw the expected convergence: 0.9 actions
    \draw[dashed,-,yshift=0](0,9) -- (1000,9);

\end{tikzpicture}

}
\qquad
\subfigure[BUL]{\label{fig:result_1b}
\begin{tikzpicture}[x=0.0068cm,y=0.05cm]
    % Draw the axes and grid lines
    \draw[-] (0,0) -- (0,100) -- (1000,100) -- (1000,0) -- cycle; 
    \draw[-,thin, dotted, ystep=25, xstep=1000] (0,0) grid (1000,100);
    \foreach \x in {250, 500, 750}  \draw [-,xshift=0](\x,4pt) -- (\x,-1pt);
    \foreach \y in {0,25,50,75,100}  \draw [-,yshift=0](4pt,\y) -- (1pt,\y);
    \foreach \x/\xtext in {250, 500, 750} \node at (\x,0) [below] {$\x$};
    \foreach \y/\ytext in {0,25,50,75,100}  \node at (0,\y) [left] {$\y$};
    \node at (0,115) {Actions};
    \node at (900,20) {Episodes};
    \draw[-] plot[mark=x,mark size=4,mark options={color=black}] 
			file {Data/testfr01na.SP.tikzdata};
    \draw[-] plot[mark=o,mark size=4,mark options={color=black}] 
			file {Data/testfr01nb.SP.tikzdata};
    % Also draw the expected convergence: 0.9 actions
    \draw[dashed,-,yshift=0](0,9) -- (1000,9);

\end{tikzpicture}

}
\caption{Agent performance under \BUL\ and \CL\ schemes. Circles represent performance without failure recovery while crosses show the performance with failure recovery enabled.
Each point represents results from $5$ experiment runs using an averaging window
of $5$ samples. The dashed line represents optimal performance ($9$ actions). Actions are non-deterministic ($10\%$ probability of failure) and failed actions do not change the world in a negative way i.e. success is always possible with recovery.}
\end{center}
\end{figure*}

\bibliography{bdilearning}
\bibliographystyle{icml2010}

\end{document} 





